%-----------------------------------------------
%  Engineer's & Master's Thesis Template
%  Copyleft by Artur M. Brodzki & Piotr Woźniak
%  Warsaw University of Technology, 2019-2022
%-----------------------------------------------

\documentclass[
    bindingoffset=5mm,  % Binding offset
    footnoteindent=3mm, % Footnote indent
    hyphenation=true    % Hyphenation turn on/off
]{src/wut-thesis}

\graphicspath{{tex/img/}} % Katalog z obrazkami.
\addbibresource{bibliografia.bib} % Plik .bib z bibliografią

%-------------------------------------------------------------
% Wybór wydziału:
%  \facultyeiti: Wydział Elektroniki i Technik Informacyjnych
%  \facultymeil: Wydział Mechaniczny Energetyki i Lotnictwa
% --
% Rodzaj pracy: \EngineerThesis, \MasterThesis
% --
% Wybór języka: \langpol, \langeng
%-------------------------------------------------------------
\facultyeiti    % Wydział Elektroniki i Technik Informacyjnych
\MasterThesis % Praca inżynierska
\langpol % Praca w języku polskim

\begin{document}

%------------------
% Strona tytułowa
%------------------
\instytut{Automatyki i Informatyki Stosowanej}
\kierunek{Informatyka}
\specjalnosc{Inteligetne Systemy}
\title{
    Porównanie wydajności i możliwości współczesnych silników gier komputerowych
}
% Title in English for English theses
% In English theses, you may remove this command
\engtitle{
    Comparison of performance and capabilities of modern computer games engines
}

\author{inż. Krzysztof Rudnicki}
\album{307585}
\promotor{dr inż. Michał Chwesiuk}
\date{\the\year}
\maketitle

%-------------------------------------
% Streszczenie po polsku dla \langpol
% English abstract if \langeng is set
%-------------------------------------
\cleardoublepage % Zaczynamy od nieparzystej strony
\abstract 
Niniejsza praca przedstawia kompleksowe porównanie dwóch wiodących silników gier komputerowych: Unity oraz Unreal Engine. Badania obejmują zarówno analizę ilościową (testy wydajnościowe z~wykorzystaniem NVIDIA Nsight Graphics), jak i~jakościową (wywiady z~ośmioma deweloperami gier posiadającymi praktyczne doświadczenie w~obu silnikach). W~ramach pracy zaimplementowano identyczną grę z~gatunku bullet hell w~obu środowiskach, co pozwoliło na bezpośrednie porównanie procesu deweloperskiego oraz wydajności końcowych aplikacji.

Wyniki badań wskazują, że Unity oferuje niższy próg wejścia, lepsze wsparcie dla gier 2D oraz szybszy cykl iteracji dzięki natywnej obsłudze hot reload. Unreal Engine natomiast wyróżnia się zaawansowanymi możliwościami graficznymi, systemem wizualnego programowania Blueprints oraz lepszym wsparciem dla produkcji AAA. Testy wydajnościowe wykazały różnice w~zarządzaniu pamięcią wynikające z~odmiennych podejść architektonicznych: garbage collector w~Unity (C\#) versus ręczne zarządzanie pamięcią w~Unreal (C++).

Praca dostarcza praktycznych rekomendacji dotyczących wyboru silnika w~zależności od typu projektu, doświadczenia zespołu oraz wymagań technicznych. Wyniki mogą być przydatne zarówno dla początkujących deweloperów podejmujących decyzję o~wyborze pierwszego silnika, jak i~dla doświadczonych zespołów rozważających migrację między platformami.

\keywords silnik gier, Unity, Unreal Engine, porównanie wydajności, bullet hell, profilowanie GPU, NVIDIA Nsight, tworzenie gier

%----------------------------------------
% Streszczenie po angielsku dla \langpol
% Polish abstract if \langeng is set
%----------------------------------------
\clearpage
\secondabstract 
This thesis presents a~comprehensive comparison of two leading game engines: Unity and Unreal Engine. The research encompasses both quantitative analysis (performance testing using NVIDIA Nsight Graphics) and qualitative analysis (interviews with eight game developers with practical experience in both engines). As part of the study, an identical bullet hell game was implemented in both environments, enabling direct comparison of the development process and final application performance.

The findings indicate that Unity offers a~lower entry barrier, better support for 2D games, and a~faster iteration cycle due to native hot reload support. Unreal Engine, on the other hand, excels in advanced graphical capabilities, the Blueprints visual programming system, and better support for AAA productions. Performance tests revealed differences in memory management resulting from distinct architectural approaches: garbage collector in Unity (C\#) versus manual memory management in Unreal (C++).

The thesis provides practical recommendations for engine selection depending on project type, team experience, and technical requirements. The results may be useful for both novice developers making decisions about their first engine choice and experienced teams considering migration between platforms.

\secondkeywords game engine, Unity, Unreal Engine, performance comparison, bullet hell, GPU profiling, NVIDIA Nsight, game development

\pagestyle{plain}

%--------------
% Spis treści
%--------------
\cleardoublepage % Zaczynamy od nieparzystej strony
\tableofcontents

%------------
% Rozdziały
%------------
\cleardoublepage % Zaczynamy od nieparzystej strony
\pagestyle{headings}

\clearpage % Rozdziały zaczynamy od nowej strony.
\section{Wstęp}

\subsection{Motywacja i cel pracy}
Współczesny rynek gier komputerowych charakteryzuje się dynamicznym rozwojem technologicznym i~rosnącymi wymaganiami zarówno twórców, jak i~graczy. Wybór odpowiedniego silnika gier jest kluczową decyzją, która wpływa na cały proces tworzenia gry, jej wydajność oraz możliwości techniczne.

Celem niniejszej pracy jest kompleksowe porównanie wydajności i~możliwości współczesnych silników gier komputerowych, ze szczególnym uwzględnieniem ich wpływu na proces tworzenia gier oraz końcową jakość produktu.

\subsection{Zakres pracy}
Praca obejmuje analizę następujących aspektów:
\begin{itemize}
    \item Wydajność renderowania grafiki 2D i~3D
    \item Możliwości i~funkcjonalności oferowane przez różne silniki
    \item Łatwość użycia i~krzywa uczenia się
    \item Wsparcie dla różnych platform docelowych
    \item Ekosystem narzędzi i~społeczność deweloperska
\end{itemize}

\subsection{Wybór gry testowej -- gatunek bullet hell}
\label{subsec:bullet-hell}

W~celu przeprowadzenia praktycznych testów wydajnościowych zdecydowano się na implementację gry z~gatunku \textbf{bullet hell} (dosł.~,,piekło pocisków''), znanego również jako \textbf{danmaku} (z~jap.~,,kurtyna pocisków'') lub \textbf{manic shooter}.

\subsubsection{Charakterystyka gatunku}

Bullet hell to podgatunek gier typu shoot 'em up (strzelanka), w~którym gracz steruje zwykle niewielkim statkiem kosmicznym lub postacią, mierząc się z~falami przeciwników wystrzeliwujących ogromne ilości pocisków tworzących skomplikowane wzory na ekranie. Kluczowe cechy gatunku obejmują:

\begin{itemize}
    \item \textbf{Masowa ilość pocisków} -- na ekranie jednocześnie może znajdować się od kilkuset do kilku tysięcy pocisków, tworzących złożone formacje geometryczne
    \item \textbf{Precyzyjne hitboxy} -- obszar kolizji postaci gracza jest znacznie mniejszy niż jej wizualna reprezentacja (często ograniczony do kilku pikseli), co umożliwia nawigację między gęstymi wzorami pocisków
    \item \textbf{Wzory pocisków} -- przeciwnicy wystrzeliwują pociski według określonych algorytmów, tworząc spirale, fale, rozgałęzienia i~inne formacje
    \item \textbf{Ciągły ruch} -- gracz musi nieustannie przemieszczać się po ekranie, unikając kolizji
    \item \textbf{Eskalacja trudności} -- wraz z~postępem gry wzrasta liczba przeciwników i~gęstość pocisków
\end{itemize}

Klasyczne przykłady gatunku to serie \textit{Touhou Project}, \textit{DoDonPachi}, \textit{Ikaruga} oraz \textit{Geometry Wars}.

\subsubsection{Uzasadnienie wyboru gatunku}

Gatunek bullet hell został wybrany jako podstawa testów wydajnościowych z~następujących powodów:

\begin{enumerate}
    \item \textbf{Intensywne wykorzystanie zasobów} -- jednoczesne renderowanie setek lub tysięcy obiektów (pocisków) stanowi znaczące obciążenie dla systemu renderowania
    
    \item \textbf{Testowanie zarządzania pamięcią} -- ciągłe tworzenie i~niszczenie obiektów pocisków eksponuje różnice w~implementacji garbage collectora (Unity/C\#) versus ręcznego zarządzania pamięcią (Unreal/C++)
    
    \item \textbf{Wymagania systemu fizyki} -- wykrywanie kolizji między graczem a~setkami pocisków w~każdej klatce obciąża system fizyki
    
    \item \textbf{Prostota implementacji} -- podstawowa mechanika gry jest stosunkowo prosta koncepcyjnie, co pozwala skupić się na porównaniu wydajności, a~nie złożoności logiki gry
    
    \item \textbf{Skalowalność testu} -- łatwo kontrolować poziom obciążenia poprzez modyfikację liczby aktywnych pocisków i~przeciwników
    
    \item \textbf{Reprezentatywność dla gier 2D} -- gatunek jest typowym przedstawicielem gier 2D, co pozwala ocenić wsparcie silników dla tego segmentu rynku
    
    \item \textbf{Wymuszenie optymalizacji} -- ze względu na ekstremalną liczbę obiektów, implementacja bullet hell wymusza stosowanie technik optymalizacyjnych (object pooling, spatial partitioning), których efektywność może różnić się między silnikami
\end{enumerate}

\subsubsection{Parametry gry testowej}

Zaimplementowana gra testowa charakteryzuje się następującymi parametrami:

\begin{itemize}
    \item Czas rozgrywki: 90 sekund (tryb przetrwania)
    \item Eskalacja trudności: liniowy wzrost częstotliwości spawnu przeciwników
    \item Typy przeciwników: 3 warianty z~różnymi wzorami strzelania
    \item Maksymalna liczba jednoczesnych pocisków: do 500 obiektów
    \item System punktacji oparty na eliminacji przeciwników
    \item Object pooling dla pocisków (eliminacja alokacji w~runtime)
\end{itemize}

Te parametry zapewniają wystarczające obciążenie systemu do ujawnienia różnic wydajnościowych między silnikami, pozostając jednocześnie w~granicach typowych dla gier indie z~tego gatunku.

\subsection{Struktura pracy}

Praca składa się z~następujących rozdziałów:

\begin{enumerate}
    \item \textbf{Wstęp} -- wprowadzenie do tematyki, motywacja, cel i~zakres pracy
    \item \textbf{Przegląd literatury} -- analiza istniejących badań porównawczych silników gier
    \item \textbf{Charakterystyka silników} -- szczegółowy opis Unity i~Unreal Engine
    \item \textbf{Metodologia} -- opis metodyki badawczej i~kryteriów porównania
    \item \textbf{Analiza wywiadów} -- wyniki badań jakościowych z~deweloperami
    \item \textbf{Implementacja gry testowej} -- doświadczenia z~tworzenia gry w~obu silnikach
    \item \textbf{Narzędzia profilowania} -- opis NVIDIA Nsight i~metodyki pomiarów
    \item \textbf{Testy wydajności} -- wyniki pomiarów wydajnościowych
    \item \textbf{Analiza możliwości} -- porównanie funkcjonalności silników
    \item \textbf{Porównanie wyników} -- synteza i~analiza zebranych danych
    \item \textbf{Podsumowanie} -- wnioski i~rekomendacje
\end{enumerate}

\subsection{Metodologia}

W~pracy zastosowano metodologię badawczą łączącą podejście ilościowe z~jakościowym:

\begin{itemize}
    \item \textbf{Testy wydajnościowe} -- obiektywne pomiary z~wykorzystaniem NVIDIA Nsight Graphics, zapewniające porównywalność wyników między silnikami
    \item \textbf{Wywiady z~deweloperami} -- badania jakościowe dostarczające kontekstu praktycznego użytkowania silników
    \item \textbf{Implementacja porównawcza} -- stworzenie identycznej gry w~obu silnikach, dokumentując różnice w~procesie deweloperskim
    \item \textbf{Analiza dokumentacji} -- przegląd oficjalnej dokumentacji i~materiałów edukacyjnych
\end{itemize}

Takie wieloaspektowe podejście pozwala na kompleksową ocenę silników, uwzględniającą zarówno mierzalne parametry techniczne, jak i~subiektywne doświadczenia użytkowników.
                    % Wstęp
\clearpage
\section{Przegląd literatury i istniejących rozwiązań}

\subsection{Historia rozwoju silników gier}
Silniki gier ewoluowały znacząco od prostych bibliotek graficznych lat 80. i 90. XX wieku po współczesne, kompleksowe środowiska deweloperskie. Według Ullmann et al. \cite{ullmann2022game}, współczesne silniki gier charakteryzują się modularną architekturą, która umożliwia ponowne wykorzystanie komponentów między różnymi projektami.

Gregory \cite{gregory2018game} w swojej fundamentalnej pracy "Game Engine Architecture" przedstawia kompleksowy przegląd ewolucji silników gier, definiując je jako "oprogramowanie zaprojektowane specjalnie do tworzenia gier". Jego analiza pokazuje, że współczesne silniki gier składają się z kilku kluczowych warstw: warstwy platformy (platform layer), warstwy podstawowych systemów (core systems), warstwy zasobów (resource manager), warstwy renderingu (rendering engine), systemów animacji, fizyki oraz gameplay. Ta architektura warstwowa umożliwia modularność i ponowne wykorzystanie komponentów.

Pierwsze silniki gier były ściśle powiązane z konkretnym sprzętem i grami, jak np. silniki do gier id Software (Doom, Quake). Według Gregory'ego \cite{gregory2018game}, przełomem było zrozumienie, że oddzielenie logiki gry od podstawowej infrastruktury technicznej pozwala na tworzenie bardziej uniwersalnych rozwiązań. Przełomem było wprowadzenie pierwszych uniwersalnych silników, które mogły być adaptowane do różnych rodzajów gier. Dzisiejsze silniki oferują zintegrowane środowiska deweloperskie z edytorami wizualnymi, systemami skryptowymi i zaawansowanymi narzędziami do debugowania.

\subsection{Klasyfikacja silników gier}

\subsubsection{Architektura silników według Gregory'ego}
Gregory \cite{gregory2018game} przedstawia szczegółową taksonomię architektur silników gier, wyróżniając kilka kluczowych typów organizacji:

\begin{itemize}
    \item \textbf{Silniki obiektowe} - bazujące na hierarchii obiektów gry z dziedziczeniem
    \item \textbf{Silniki komponentowe} - wykorzystujące systemy entity-component-system (ECS)
    \item \textbf{Silniki hybrydowe} - łączące elementy różnych podejść architektonicznych
\end{itemize}

Autor podkreśla, że wybór architektury ma fundamentalny wpływ na wydajność, skalowalność i łatwość rozwoju gier. Systemy ECS zyskują na popularności ze względu na lepszą wydajność cache procesora i większą elastyczność w definiowaniu zachowań obiektów gry.

\subsubsection{Silniki komercyjne vs. open source}
Analiza literatury pokazuje wyraźne różnice między rozwiązaniami komercyjnymi a otwartymi. Christopoulou i Xinogalos \cite{christopoulou2017overview} wskazują, że silniki komercyjne jak Unity czy Unreal Engine oferują lepsze wsparcie techniczne i dokumentację, podczas gdy rozwiązania open source zapewniają większą elastyczność i kontrolę nad kodem źródłowym.

Sharif i Ameen \cite{sharif2021game} podkreślają, że wybór między rozwiązaniem komercyjnym a open source zależy głównie od budżetu projektu i wymagań dotyczących dostosowania silnika do specyficznych potrzeb.

\subsubsection{Silniki specjalistyczne vs. uniwersalne}
Pavkov et al. \cite{pavkov2017comparison} przedstawiają podział na silniki dedykowane konkretnym gatunkom gier (np. silniki do gier strategicznych czasu rzeczywistego) oraz rozwiązania uniwersalne mogące obsługiwać różnorodne typy gier. Silniki specjalistyczne oferują zoptymalizowane funkcjonalności dla określonego zastosowania, podczas gdy uniwersalne zapewniają większą wszechstronność kosztem specjalizacji.

\subsection{Aktualny stan badań}

\subsubsection{Badania wydajności}
Messaoudi et al. \cite{messaoudi2017performance} przeprowadzili kompleksową analizę wydajności silnika Unity na urządzeniach mobilnych i stacjonarnych, koncentrując się na zużyciu CPU i optymalizacji logiki gry. Ich badania pokazują znaczące różnice w wydajności między platformami mobilnymi a desktop.

Abramowicz i Borczuk \cite{abramowicz2024comparative} porównali wydajność Unity i Unreal Engine w grach 3D, skupiając się na renderowaniu, systemach fizyki i zarządzaniu pamięcią. Wyniki wskazują na przewagę Unreal Engine w renderowaniu zaawansowanej grafiki 3D, podczas gdy Unity wykazuje lepszą wydajność na urządzeniach o ograniczonych zasobach.

\subsubsection{Metodologie porównawcze}
Pattrasitidecha \cite{pattrasitidecha2014comparison} opracował macierz porównawczą dla silników gier mobilnych 3D, definiując kryteria selekcji i kluczowe aspekty oceny. Ta metodologia została szeroko adoptowana w późniejszych badaniach.

Vohera et al. \cite{vohera2021game} przedstawili architekturę silników gier i przeprowadzili studium porównawcze Unity, GameMaker, Unreal Engine i CryEngine, koncentrując się na parametrach wydajności, funkcjonalności i łatwości użycia.

\subsubsection{Specjalizowane zastosowania}
Marks et al. \cite{marks2008evaluation} oceniali silniki gier pod kątem zastosowań w symulacjach medycznych i szkoleniach klinicznych, wprowadzając specyficzne kryteria oceny dla aplikacji edukacyjnych.

Ali i Usman \cite{ali2016framework} opracowali framework do selekcji silników gier dla zastosowań w gamifikacji i serious games, uwzględniając specyficzne wymagania tych dziedzin.

\subsubsection{Badania społeczności i ekosystemu}
Barczak i Woźniak \cite{barczak2019comparative} przeprowadzili kompleksowe studium porównawcze silników gier, analizując nie tylko aspekty techniczne, ale również dostępność zasobów edukacyjnych, aktywność społeczności i długoterminowe wsparcie.

\subsection{Identyfikacja luk badawczych}

Analiza dostępnej literatury ujawnia kilka istotnych luk badawczych:

\begin{enumerate}
    \item \textbf{Brak kompleksowych badań wielokryterialnych} - większość istniejących prac koncentruje się na pojedynczych aspektach (wydajność, funkcjonalność) bez holistycznego podejścia
    
    \item \textbf{Ograniczone badania długoterminowe} - brakuje analiz wpływu aktualizacji silników na stabilność i wydajność projektów
    
    \item \textbf{Niewystarczające dane o współczesnych silnikach} - wiele badań koncentruje się na starszych wersjach silników, nie uwzględniając najnowszych możliwości
    
    \item \textbf{Brak standaryzacji metodologii} - różne badania stosują odmienne kryteria oceny, co utrudnia porównanie wyników
    
    \item \textbf{Ograniczone badania cross-platform} - niewiele prac analizuje wydajność silników na różnych platformach docelowych w sposób systematyczny
\end{enumerate}

Niniejsza praca ma na celu wypełnienie tych luk poprzez przeprowadzenie kompleksowej analizy porównawczej współczesnych silników gier z zastosowaniem ustandaryzowanej metodologii i wielokryterialnego podejścia do oceny.

\subsection{Trendy technologiczne}
Ostatnie badania wskazują na rosnące znaczenie technologii ray tracing, sztucznej inteligencji w grach oraz wsparcia dla rzeczywistości wirtualnej i rozszerzonej. Masood et al. \cite{masood2022high} analizują wykorzystanie silników gier do wysokowydajnego renderowania terenu GPU, pokazując nowe kierunki rozwoju technologii renderowania.

Badania Firat et al. \cite{firat2022sound} dotyczące przestrzennego dźwięku 3D w silnikach gier wskazują na rosnące znaczenie immersyjnych doświadczeń audio jako czynnika różnicującego poszczególne rozwiązania.

% Bibliografia zawiera odniesienia do kluczowych publikacji naukowych,
% dokumentacji technicznej oraz raportów branżowych
       % Przegląd literatury i istniejących rozwiązań
\clearpage
\section{Charakterystyka współczesnych silników gier}

\subsection{Kryteria wyboru silników do analizy}
W ramach niniejszej pracy wybrano reprezentatywne silniki gier reprezentujące różne segmenty rynku:
\begin{itemize}
    \item Popularność i rozpowszechnienie
    \item Różnorodność technologiczna
    \item Wsparcie dla różnych platform
    \item Dostępność dokumentacji i narzędzi
\end{itemize}

\subsection{Unity}
\subsubsection{Architektura i technologie}
% Opis architektury silnika Unity, stosowanych technologii

\subsubsection{Możliwości i funkcjonalności}
% Szczegółowy opis oferowanych funkcji

\subsubsection{Narzędzia deweloperskie}
% Edytor, debugger, profiler, etc.

\subsection{Unreal Engine}
\subsubsection{Architektura i technologie}
% Opis architektury Unreal Engine

\subsubsection{Możliwości i funkcjonalności}
% Blueprint system, C++ support, rendering pipeline

\subsubsection{Narzędzia deweloperskie}
% Unreal Editor, Visual Scripting, etc.

\subsection{Godot}
\subsubsection{Architektura i technologie}
% Opis architektury silnika Godot

\subsubsection{Możliwości i funkcjonalności}
% GDScript, scene system, node-based architecture

\subsubsection{Narzędzia deweloperskie}
% Integrated editor, scripting tools

\subsection{Inne analizowane silniki}
% Krótka charakterystyka dodatkowych silników (np. CryEngine, GameMaker Studio)

\subsection{Porównanie tabelaryczne podstawowych cech}
% Tabela porównawcza kluczowych parametrów
              % Charakterystyka współczesnych silników gier
\clearpage
\section{Metodologia badań i kryteria porównania}

\subsection{Założenia metodologiczne}
\subsubsection{Cel badań}
Głównym celem badań jest obiektywne porównanie wydajności i możliwości wybranych silników gier w kontrolowanych warunkach.

\subsubsection{Hipotezy badawcze}
\begin{enumerate}
    \item Silniki komercyjne oferują lepszą wydajność niż rozwiązania open source
    \item Kompleksowość funkcjonalności wpływa negatywnie na wydajność
    \item Łatwość użycia jest odwrotnie proporcjonalna do możliwości konfiguracji
\end{enumerate}

\subsection{Kryteria porównania}
\subsubsection{Wydajność}
\begin{itemize}
    \item Szybkość renderowania (FPS)
    \item Zużycie pamięci RAM
    \item Obciążenie procesora
    \item Zużycie pamięci karty graficznej
    \item Czas ładowania scen
\end{itemize}

\subsubsection{Funkcjonalność}
\begin{itemize}
    \item Wsparcie dla różnych typów renderingu
    \item Systemy fizyki
    \item Systemy audio
    \item Wsparcie dla VR/AR
    \item Możliwości skryptowania
\end{itemize}

\subsubsection{Użyteczność}
\begin{itemize}
    \item Intuicyjność interfejsu
    \item Jakość dokumentacji
    \item Dostępność tutoriali
    \item Wsparcie społeczności
    \item Czas potrzebny na naukę
\end{itemize}

\subsection{Środowisko testowe}
\subsubsection{Specyfikacja sprzętowa}
% Szczegółowy opis konfiguracji sprzętowej używanej do testów

\subsubsection{Specyfikacja oprogramowania}
% Wersje systemów operacyjnych, sterowników, silników

\subsection{Projekt testów}
\subsubsection{Scenariusze testowe}
% Opis konkretnych scenariuszy używanych do porównania

\subsubsection{Metryki i wskaźniki}
% Definicja mierzonych parametrów i sposobów ich pomiaru
               % Metodologia badań i kryteria porównania
\clearpage
\section{Analiza wywiadów z deweloperami gier}
\label{sec:wywiady}

W ramach badań jakościowych przeprowadzono osiem pogłębionych wywiadów z deweloperami gier posiadającymi doświadczenie w pracy z silnikami Unity i Unreal Engine. Celem badania było zebranie praktycznych spostrzeżeń dotyczących użyteczności, wydajności oraz przepływu pracy w obu silnikach z perspektywy osób aktywnie je wykorzystujących.

\subsection{Charakterystyka respondentów}
\label{subsec:charakterystyka-respondentow}

Respondenci zostali dobrani według kryterium posiadania co najmniej rocznego doświadczenia amatorskiego lub profesjonalnego w jednym z badanych silników. Profil uczestników przedstawia się następująco:

\begin{itemize}
    \item \textbf{Respondent 1}: Około 6-10 lat doświadczenia amatorskiego w Unity, semestr zajęć z Unreal Engine, 10-20 projektów w Unity
    \item \textbf{Respondent 2}: 7 lat doświadczenia amatorskiego w Unity, pół roku profesjonalnego, 15-20 projektów
    \item \textbf{Respondent 3}: 1,5 roku amatorskiego doświadczenia w Unity, 4 projekty zakończone
    \item \textbf{Respondent 4}: 2 lata profesjonalne w Unreal, 2 miesiące w Unity (z przerwami przez kilka lat), projekty w obu silnikach
    \item \textbf{Respondent 5}: 9 lat doświadczenia zawodowego (od 2012 Unity amatorsko, od 2016 profesjonalnie; od 2019 Unreal profesjonalnie), 10-30 projektów w Unity, 5-6 w Unreal
    \item \textbf{Respondent 6}: Dekada doświadczenia amatorskiego w Unity, kilka projektów game jamowych
    \item \textbf{Respondent 7}: 9 lat hobbystycznego doświadczenia w Unity, 2 lata profesjonalnego; 1-1,5 roku amatorskiego w Unreal
    \item \textbf{Respondent 8}: 2 lata amatorsko w Unity, 1,5 roku profesjonalnie + pół roku stażu w Unreal, kilkanaście projektów w obu silnikach
\end{itemize}

Łącznie badana próba reprezentuje szerokie spektrum doświadczeń -- od osób skupionych wyłącznie na Unity, przez deweloperów wykorzystujących oba silniki, po profesjonalistów pracujących głównie w Unreal Engine.

\subsection{Motywy wyboru silnika}
\label{subsec:motywy-wyboru}

\subsubsection{Przystępność i próg wejścia}

Dominującym motywem wyboru Unity jako pierwszego silnika była jego \textbf{przystępność dla początkujących}. Respondenci wskazywali, że Unity oferuje mniejszą liczbę gotowych mechanik widocznych na starcie -- silnik nie narzuca użytkownikowi wbudowanych rozwiązań, jeżeli ten nie wybierze specjalnego szablonu projektu. Było to postrzegane jako zaleta dydaktyczna, ponieważ nowicjusze nie byli przytłaczani złożonością interfejsu.

Jednocześnie respondenci podkreślali, że Unreal Engine w przeszłości (około 2018 roku) charakteryzował się znacznie wyższym progiem wejścia niż obecnie. W tamtym okresie dostępnych było również więcej materiałów edukacyjnych dla Unity, co dodatkowo wpływało na wybór tego silnika przez początkujących.

Paradoksalnie, mniejsza liczba wbudowanych funkcjonalności w Unity była postrzegana jako zaleta dydaktyczna -- silnik nie przytłaczał nowicjuszy złożonością interfejsu i pozwalał na stopniowe poznawanie kolejnych mechanizmów.

\subsubsection{Język programowania}

Wybór C\# jako głównego języka skryptowania w Unity stanowił istotny czynnik decyzyjny dla osób z wcześniejszym doświadczeniem w tym języku. Respondenci z backgroundem w C\# określali przejście do Unity jako naturalne i intuicyjne. Język ten był opisywany jako wysokopoziomowy, niewymagający ręcznego zarządzania pamięcią, co znacząco obniża barierę wejścia dla początkujących programistów.

Niektórzy respondenci zwracali uwagę, że C++ używany w Unreal Engine różni się od standardowego C++ -- jest rozszerzony o makra i mechanizmy specyficzne dla silnika, co może być zaskakujące dla programistów przyzwyczajonych do klasycznego C++.

\subsubsection{Wymagania projektu}

Wybór Unreal Engine często był podyktowany specyfiką projektu lub wymaganiami rynku pracy. Respondenci wskazywali, że projekty wymagające wysokiej jakości grafiki naturalnie kierowały ich w stronę Unreal Engine. Dodatkowo, część osób rozpoczęła naukę Unreal ze względu na większą liczbę ofert pracy wymagających znajomości tego silnika, szczególnie w segmencie gier AAA i większych studiów deweloperskich.

\subsection{Dokumentacja i materiały edukacyjne}
\label{subsec:dokumentacja}

\subsubsection{Oficjalna dokumentacja}

W zakresie dokumentacji oficjalnej respondenci wyraźnie faworyzowali Unity. Dokumentacja tego silnika była opisywana jako dogłębna i szczegółowa -- praktycznie wszystkie klasy, metody i właściwości są dokładnie opisane, a dodatkowo często zawierają działające przykłady kodu, które można bezpośrednio skopiować i uruchomić w projekcie.

Dokumentacja Unreal Engine była oceniana znacznie gorzej. Respondenci określali ją jako szkieletową lub wręcz nieistniejącą w praktycznym sensie. Wiele stron dokumentacji zawiera jedynie nazwę funkcji i nazwy parametrów, bez jakiegokolwiek opisu działania. Jeden z respondentów porównał czytanie dokumentacji Unreal do przeglądania plików nagłówkowych (header files), gdzie użytkownik musi samodzielnie domyślać się, co dana funkcja robi.

Jako pozytywny aspekt ekosystemu Unreal wskazywano fora deweloperskie, gdzie profesjonalni użytkownicy dzielą się rozwiązaniami. Problemem jest jednak to, że część najbardziej wartościowych zasobów znajduje się w zamkniętych sekcjach forum, dostępnych tylko dla wybranych firm po uzyskaniu specjalnych uprawnień od Epic Games.

\subsubsection{Nieoficjalne poradniki}

W przypadku materiałów nieoficjalnych (YouTube, blogi, fora) Unity również dominowało ilościowo. Respondenci szczególnie wyróżniali kanał Brackeys jako kluczowe źródło wiedzy dla początkujących i średniozaawansowanych użytkowników Unity.

Poradniki do Unreal Engine były oceniane jako:
\begin{itemize}
    \item Mniej liczne niż dla Unity
    \item Często nieaktualne -- dotyczące starszych wersji silnika (np. Unreal 4), które mogą, ale nie muszą działać w nowszych wersjach
    \item Zbyt skoncentrowane na systemie Blueprints kosztem programowania w C++
\end{itemize}

\subsubsection{Jakość dydaktyczna poradników}

Respondenci zwracali uwagę na wspólny problem poradników do obu silników -- koncentrację na implementacji konkretnych funkcji kosztem dobrych praktyk programistycznych. Większość dostępnych materiałów skupia się na pokazaniu, jak zaimplementować pojedynczą mechanikę, bez wyjaśniania szerszego kontekstu architektonicznego czy zasad rozszerzalności kodu.

Ten brak holistycznego podejścia sprawia, że początkujący deweloperzy potrafią zaimplementować poszczególne funkcje, ale mają trudności z połączeniem ich w spójną całość lub późniejszym rozwojem projektu.

\subsection{Architektura i wzorce projektowe}
\label{subsec:architektura}

\subsubsection{System komponentowy Unity}

Architektura Unity oparta na komponentach była oceniana pozytywnie pod względem elastyczności. Respondenci doceniali możliwość dzielenia funkcjonalności na małe, niezależne moduły (komponenty), które następnie można łączyć w większe całości.

Jednocześnie wskazywano na problemy wynikające z długu technologicznego Unity. Silnik jest bardzo monolityczny, z głęboką hierarchią dziedziczenia podstawowych konceptów. Niektóre obiekty bazowe zajmują tak dużo pamięci, że nie mieszczą się w pojedynczej linii cache procesora, co na współczesny hardware stanowi istotny problem wydajnościowy.

\subsubsection{Struktura Unreal Engine}

Architektura Unreal Engine wymusza bardziej uporządkowany styl pracy. Respondenci zauważali, że nawet podstawowe projekty tworzone w Unreal mają tendencję do bycia lepiej zorganizowanymi, ponieważ silnik narzuca określoną strukturę.

Struktura aktor-komponent w Unreal (level zawiera aktorów, aktorzy zawierają komponenty) została opisana jako bardziej restrykcyjna niż prefaby w Unity. Próby tworzenia zagnieżdżonych struktur (aktor w aktorze) często prowadzą do problemów, podczas gdy w Unity hierarchie prefabów są bardziej elastyczne.

\subsubsection{Specjalizacja silników}

Respondenci zauważyli, że Unreal Engine jest wyraźnie zoptymalizowany pod gry typu first-person shooter. Tworzenie gier FPS w Unreal jest niezwykle proste -- wystarczy zaznaczyć odpowiednie opcje. Natomiast projekty odbiegające od tego wzorca (np. gry z rozbudowanym interfejsem użytkownika, gry turowe) wymagają znacznie więcej pracy i często sprowadzają się do obchodzenia domyślnych mechanizmów silnika.

\subsection{Kompilacja i przepływ pracy}
\label{subsec:kompilacja}

\subsubsection{Czas kompilacji}

Czas kompilacji w Unity był identyfikowany jako znaczący problem przy większych projektach. W miarę rozrastania się bazy kodu, czas potrzebny na rekompilację po każdej zmianie rośnie.

Unity oferuje mechanizm Assembly Definitions jako rozwiązanie tego problemu. Bez podziału projektu na osobne assemblies każda zmiana w kodzie powoduje rekompilację całego projektu. Podział na mniejsze moduły pozwala kompilować tylko zmienione fragmenty, znacząco skracając czas iteracji.

\subsubsection{Stabilność środowiska}

Istotną różnicą między silnikami jest obsługa błędów krytycznych. W Unity gra uruchomiona w edytorze działa jako osobny proces -- gdy wystąpi błąd krytyczny, zamyka się tylko ten proces, a edytor pozostaje stabilny. W Unreal Engine silnik i gra działają jako jeden proces, więc crash w grze powoduje utratę całego edytora wraz z ewentualnymi niezapisanymi zmianami.

Ta różnica architekturalna ma istotne konsekwencje dla produktywności, szczególnie przy debugowaniu. Przy dużych projektach, gdzie uruchomienie silnika może trwać kilkanaście minut, każdy crash oznacza znaczną stratę czasu.

\subsubsection{Kompatybilność wsteczna}

Unreal Engine był krytykowany za problemy z kompatybilnością między wersjami. Respondenci wskazywali, że rozpoczęcie projektu w określonej wersji silnika może skutkować problemami, jeśli ta wersja okaże się zawierać fundamentalne błędy. Epic Games nie backportuje poprawek do starszych wersji w takim stopniu jak Unity robi to dla wersji LTS.

\subsection{Kontrola wersji i współpraca zespołowa}
\label{subsec:kontrola-wersji}

\subsubsection{Integracja z Git}

Współpraca z systemem Git była oceniana lepiej dla Unity ze względu na tekstową serializację assetów. Pliki scen i prefabów w Unity są zapisywane w formacie YAML, co teoretycznie umożliwia ich mergowanie. Nowoczesne narzędzia (np. merge w Rider) potrafią automatycznie rozwiązywać niektóre konflikty na scenach.

Pliki binarne w Unreal Engine stanowią znaczące wyzwanie. Respondenci zwracali uwagę, że nawet pliki Blueprintów, które ewidentnie mają serializację tekstową, są zapisywane na dysku jako binaria. To znacznie utrudnia współpracę wielu programistów nad tym samym projektem.

\subsubsection{Mergowanie konfliktów}

Konflikty na scenach i prefabach stanowią problem w obu silnikach. Gdy dwie osoby edytują tę samą scenę, rozwiązanie konfliktu często sprowadza się do wybrania jednej wersji i ręcznego przeniesienia zmian z drugiej.

Jako rozwiązanie wskazywano praktykę lockowania plików (preferowana przy użyciu Perforce) lub podział pracy na oddzielne sceny, gdzie każdy deweloper pracuje we własnym środowisku. Unity ułatwia takie podejście dzięki elastycznemu systemowi scen, podczas gdy Unreal silniej promuje architekturę z jedną główną sceną.

\subsection{Współpraca z osobami nietechnicznymi}
\label{subsec:wspolpraca-nietechniczna}

\subsubsection{System Blueprints}

Blueprinty w Unreal Engine były postrzegane jako skuteczne narzędzie ułatwiające współpracę z osobami nietechnicznymi. System wizualnego programowania pozwala designerom i artystom na tworzenie logiki gry bez pisania kodu tekstowego. Respondenci zauważali, że osoby niebędące programistami często nie zdają sobie sprawy, że faktycznie programują, korzystając z Blueprintów.

Jednocześnie integracja Blueprintów z kodem C++ nie jest idealna. Przejście między oboma systemami wymaga dodatkowej pracy, a wystawianie funkcji C++ do Blueprintów nie zawsze działa bezproblemowo.

\subsubsection{Narzędzia dla artystów}

Unity wymaga więcej pracy przy tworzeniu narzędzi dla osób nietechnicznych. Respondenci wskazywali, że w Unreal Engine osoby nietechniczne mają lepsze wsparcie ,,out of the box'', podczas gdy w Unity zazwyczaj trzeba przeprowadzać szkolenia lub tworzyć dedykowane narzędzia edytorowe, aby umożliwić artystom i designerom samodzielną pracę.

\subsection{Asset Store i zasoby zewnętrzne}
\label{subsec:asset-store}

\subsubsection{Dostępność i jakość assetów}

Asset Store Unity był oceniany jako lepiej zarządzany i bogatszy. Respondenci wskazywali na silniejsze wsparcie społeczności i większe szanse na znalezienie potrzebnych zasobów.

Interesującą obserwacją było to, że najlepsze produkty z Asset Store mają tendencję do opuszczania platformy -- twórcy zakładają własne strony internetowe po osiągnięciu określonego poziomu popularności.

Unreal Marketplace przeszedł niedawno transformację w platformę Fab, co według respondentów pogorszyło doświadczenie użytkownika i zwiększyło liczbę kroków potrzebnych do pobrania darmowych zasobów.

\subsubsection{Zastosowanie assetów}

Assety były rekomendowane głównie do prototypowania, nie do produkcji komercyjnej. Respondenci podkreślali, że niespójny styl graficzny wynikający z łączenia assetów od różnych twórców jest gorszy niż jednolity, nawet jeśli prosty styl graficzny.

\subsection{Wykorzystanie sztucznej inteligencji}
\label{subsec:ai}

\subsubsection{Doświadczenia z LLM}

Większość respondentów miała ograniczone doświadczenia z wykorzystaniem AI w pracy z silnikami gier. Główna obserwacja dotyczyła niskiej jakości generowanego kodu -- naprawianie błędów w kodzie wygenerowanym przez ChatGPT często zajmowało więcej czasu niż napisanie rozwiązania od podstaw.

Jednocześnie AI było wykorzystywane skuteczniej jako substytut dokumentacji dla Unreal Engine. Pomimo częstych konfabulacji, modele językowe potrafiły naprowadzić na właściwe słowa kluczowe lub nazwy funkcji, które następnie można było zweryfikować w kodzie źródłowym silnika.

\subsubsection{Generowanie grafik}

Pozytywne doświadczenia zgłoszono w zakresie generowania placeholderów graficznych podczas game jamów. AI pozwala szybko uzyskać przyzwoicie wyglądające grafiki do prototypów, choć do wersji finalnych produktów nadal preferowana jest praca profesjonalnych grafików.

\subsection{Optymalizacja i wydajność}
\label{subsec:optymalizacja}

\subsubsection{Narzut silników}

Respondenci wskazywali, że Unity ma mniejszy narzut wydajnościowy niż Unreal dla prostych projektów. Czas ładowania projektów w Unity jest znacznie krótszy, co respondenci przypisywali domyślnie niższym rozdzielczościom tekstur i prostszym ustawieniom graficznym.

\subsubsection{Blueprinty vs C++}

Istotną różnicę wydajnościową w Unreal stanowi wybór między Blueprintami a kodem C++. Blueprinty są interpretowane w czasie wykonania jako dane, a nie kompilowane do kodu maszynowego. W praktyce oznacza to, że logika napisana w Blueprintach jest znacznie wolniejsza niż równoważny kod C++.

\subsubsection{Garbage Collector}

Problem garbage collectora w Unity był wielokrotnie wspominany jako znany problem, przed którym ostrzegają doświadczeni deweloperzy. Cykliczne uruchamianie garbage collectora może powodować zauważalne zacięcia w grze. Co ciekawe, wielu respondentów wspominało o tym problemie jako o teoretycznym zagrożeniu, nie mając bezpośrednich negatywnych doświadczeń -- prawdopodobnie dzięki stosowaniu praktyk takich jak object pooling.

\subsection{Przyszłość silników i oczekiwania deweloperów}
\label{subsec:przyszlosc}

\subsubsection{Entity Component System (ECS)}

Nowy system DOTS/ECS w Unity był oczekiwaną funkcjonalnością, która w momencie przeprowadzania wywiadów została już oficjalnie wydana. System ten pozwala na pisanie wysoce wydajnego, zorientowanego na dane kodu, kosztem większej złożoności programistycznej.

\subsubsection{UI Toolkit}

Nowy system UI w Unity (UI Toolkit) był wskazywany jako obszar wymagający poprawy. Respondenci wyrażali nadzieję na jego dalszy rozwój w kierunku zbliżonym do technologii frontendowych, co ułatwiłoby pracę osobom z doświadczeniem w tworzeniu aplikacji webowych.

\subsubsection{Konkurencja Godot}

Część respondentów wyraziła zainteresowanie silnikiem Godot jako alternatywą dla Unity i Unreal. Główne przyczyny to:
\begin{itemize}
    \item Model licencyjny royalty-free (brak opłat od przychodów)
    \item Otwarte źródła umożliwiające modyfikację silnika
    \item Mniejsza złożoność ułatwiająca naukę
    \item Kontrowersje związane z próbą zmiany modelu licencyjnego Unity w 2023 roku
\end{itemize}

Respondenci przewidywali, że jeśli Unity nie poprawi swojego wizerunku i oferty, Godot może w przyszłości stać się poważną konkurencją w segmencie gier indie.

\subsection{Podsumowanie wyników badań jakościowych}
\label{subsec:podsumowanie-wywiady}

Na podstawie przeprowadzonych wywiadów można sformułować następujące wnioski:

\subsubsection{Silne strony Unity}
\begin{itemize}
    \item Wysoka jakość oficjalnej dokumentacji
    \item Bogaty ekosystem materiałów edukacyjnych
    \item Niższy próg wejścia dla początkujących
    \item Lepsza integracja z systemami kontroli wersji (tekstowa serializacja)
    \item Przystępny język programowania (C\#)
    \item Elastyczna architektura komponentowa
    \item Mniejszy narzut wydajnościowy dla prostych projektów
\end{itemize}

\subsubsection{Silne strony Unreal Engine}
\begin{itemize}
    \item Wymuszona struktura projektu promująca dobre praktyki
    \item System Blueprints ułatwiający współpracę z osobami nietechnicznymi
    \item Więcej gotowych funkcjonalności ,,out of the box''
    \item Lepsze wsparcie dla projektów wysokobudżetowych (grafika, multiplayer)
    \item Dostęp do kodu źródłowego silnika
    \item Lepsza integracja z zewnętrznymi narzędziami graficznymi (np. Blender)
\end{itemize}

\subsubsection{Obszary problemowe wspólne}
\begin{itemize}
    \item Trudności z mergowaniem assetów graficznych w systemach kontroli wersji
    \item Poradniki koncentrujące się na implementacji kosztem dobrych praktyk
    \item Problemy z kompatybilnością między wersjami silników
\end{itemize}

\subsubsection{Rekomendacje z badań}

Na podstawie wywiadów można zasugerować następujące kryteria wyboru silnika:

\begin{table}[ht]
\centering
\caption{Rekomendacje wyboru silnika w zależności od kontekstu projektu}
\label{tab:rekomendacje-silnikow}
\begin{tabular}{|>{\raggedright\arraybackslash}p{4.5cm}|>{\raggedright\arraybackslash}p{4cm}|>{\raggedright\arraybackslash}p{4cm}|}
\hline
\textbf{Kryterium} & \textbf{Unity} & \textbf{Unreal Engine} \\
\hline
Doświadczenie zespołu & Początkujący, znajomość C\# & Średniozaawansowany, znajomość C++ \\
\hline
Typ projektu & Gry mobilne, 2D, indie & FPS, AAA, realistyczna grafika \\
\hline
Skład zespołu & Programiści & Mieszany (designerzy, artyści) \\
\hline
Budżet czasowy na naukę & Krótki & Średni do długiego \\
\hline
Wymagania graficzne & Standardowe & Wysokie \\
\hline
\end{tabular}
\end{table}

Wyniki badań jakościowych uzupełniają obiektywne testy wydajnościowe przedstawione w rozdziale \ref{sec:testy-wydajnosci}, dostarczając kontekstu praktycznego użytkowania obu silników w rzeczywistych projektach.
             % Analiza wywiadów z deweloperami gier
\clearpage
\section{Doświadczenia z implementacji gry testowej}
\label{sec:implementacja-gry}

W ramach praktycznej części badań zaimplementowano grę typu bullet-hell w obu porównywanych silnikach. Gatunek ten został wybrany ze względu na jego wymagania wydajnościowe -- jednoczesne renderowanie setek pocisków na ekranie stanowi doskonały test możliwości graficznych oraz efektywności zarządzania pamięcią przez silnik.

\subsection{Opis projektu testowego}

Zaimplementowana gra to klasyczny przedstawiciel gatunku bullet-hell, w~którym gracz steruje statkiem kosmicznym i~musi przetrwać przez określony czas (90~sekund), unikając pocisków wrogów i~eliminując przeciwników. Kluczowe mechaniki gry obejmują:

\begin{itemize}
    \item System spawnu wrogów z eskalującą trudnością -- częstotliwość pojawiania się przeciwników wzrasta wraz z upływem czasu
    \item System pocisków z object poolingiem -- optymalizacja pozwalająca na obsługę setek aktywnych pocisków
    \item System zdrowia i kolizji dla gracza oraz przeciwników
    \item Dynamiczne tło z efektem paralaksy
    \item System punktacji i warunki zwycięstwa/porażki
\end{itemize}

\subsection{Implementacja w Unity}
\label{subsec:impl-unity}

\subsubsection{Środowisko i konfiguracja projektu}

Projekt Unity został utworzony w~wersji LTS z~wykorzystaniem standardowego renderera 2D. Instalacja silnika na systemie Linux przebiegła bezproblemowo dzięki Unity Hub, który zapewnia spójne zarządzanie wersjami edytora i~projektami.

Struktura projektu została zorganizowana według wzorca przestrzeni nazw, co pozwoliło na czytelną organizację kodu i~uniknięcie konfliktów nazw.

\subsubsection{Architektura systemu}

Implementacja Unity wykorzystuje kilka kluczowych wzorców projektowych:

\paragraph{Wzorzec Bootstrap}
Klasa \texttt{GameBootstrap} wykorzystuje atrybut \texttt{[Runtime\-Initialize\-OnLoad\-Method]} do zapewnienia, że obiekt \texttt{GameInitializer} istnieje w~scenie przed rozpoczęciem gry. Jest to eleganckie rozwiązanie problemu inicjalizacji singletonów w~Unity.

\paragraph{Object Pooling}
System \texttt{BulletPool} stanowi rdzeń optymalizacji wydajnościowej. Zamiast ciągłego tworzenia i niszczenia obiektów pocisków (co generowałoby znaczące obciążenie garbage collectora), pociski są recyklingowane z puli:

\begin{lstlisting}[language=C, caption={Fragment implementacji object poolingu w Unity}, label={lst:unity-pool}]
public Bullet Spawn(Vector2 position, Vector2 direction, 
                    float speed, float damage)
{
    Bullet bullet = _pool.Count > 0 
        ? _pool.Dequeue() 
        : Bullet.Create(this, bulletColor, faction);
    _liveBullets.Add(bullet);
    bullet.gameObject.SetActive(true);
    bullet.transform.position = position;
    bullet.Configure(direction, speed, damage, faction);
    return bullet;
}
\end{lstlisting}

Pula jest wstępnie rozgrzewana (\textit{warm capacity}) podczas inicjalizacji, co eliminuje alokacje podczas rozgrywki.

\paragraph{Singleton Pattern}
Klasy \texttt{GameDirector} i \texttt{EnemySpawner} wykorzystują wzorzec Singleton z właściwością \texttt{Instance}, zapewniając globalny punkt dostępu do kluczowych systemów gry.

\subsubsection{System spawnu przeciwników}

\texttt{EnemySpawner} implementuje system eskalującej trudności poprzez interpolację czasu między spawnami:

\begin{lstlisting}[language=C, caption={Interpolacja trudności w Unity}, label={lst:unity-difficulty}]
float t = _elapsed / totalDuration;
float delay = Mathf.Lerp(spawnDelayStart, spawnDelayEnd, t);
\end{lstlisting}

Przeciwnicy są definiowani przez strukturę \texttt{EnemyBlueprint}, która zawiera parametry takie jak prędkość, zdrowie, wzorce strzelania i zachowania. To podejście data-driven pozwala na łatwe tworzenie różnorodnych typów wrogów.

\subsubsection{Wyzwania napotkane w Unity}

Podczas implementacji napotkano następujące wyzwania:

\begin{enumerate}
    \item \textbf{Garbage Collection} -- początkowa implementacja bez object poolingu powodowała zauważalne spadki klatek przy dużej liczbie pocisków
    \item \textbf{Kolejność inicjalizacji} -- konieczność użycia wzorca Bootstrap wynikała z~nieprzewidywalnej kolejności wywoływania metod \texttt{Awake()} i~\texttt{Start()}
    \item \textbf{Serializacja} -- atrybuty \texttt{[SerializeField]} wymagały starannego rozplanowania, które pola powinny być edytowalne w~inspektorze
\end{enumerate}

\subsubsection{Pozytywne aspekty Unity}

\begin{itemize}
    \item Natywne wsparcie dla 2D -- dedykowany tryb 2D z odpowiednimi komponentami fizyki (\texttt{Rigidbody2D}, \texttt{Collider2D})
    \item Hot reload -- możliwość edycji kodu i natychmiastowego testowania zmian
    \item Intuicyjny inspektor -- łatwa konfiguracja parametrów gry bez rekompilacji
    \item Bogata dokumentacja C\# i społeczność
\end{itemize}

\subsection{Implementacja w Unreal Engine}
\label{subsec:impl-unreal}

\subsubsection{Środowisko i konfiguracja projektu}

Instalacja Unreal Engine na systemie Linux okazała się znacznie bardziej skomplikowana niż w przypadku Unity. Dostępne są dwie ścieżki:

\begin{enumerate}
    \item Uzyskanie dostępu do oficjalnego repozytorium GitHub Epic Games i samodzielna kompilacja silnika ze źródeł
    \item Pobranie prekompilowanej wersji binarnej
\end{enumerate}

Należy zauważyć, że Unreal Engine nie oferuje wersji LTS (Long Term Support), co może stanowić wyzwanie dla długoterminowych projektów.

\subsubsection{Podejście do grafiki 2D}

Fundamentalna różnica między Unity a~Unreal w~kontekście gier 2D polega na tym, że Unreal traktuje 2D jako ,,fałszywe 2D'' -- w~rzeczywistości jest to scena 3D z~zablokowaną trzecią osią i~kamerą ortograficzną. Unity natomiast oferuje dedykowany tryb 2D z~wyspecjalizowanymi komponentami.

Ta różnica ma praktyczne konsekwencje:
\begin{itemize}
    \item W Unreal konieczne jest ręczne konfigurowanie kamery ortograficznej
    \item Fizyka 2D w~Unreal wykorzystuje te same komponenty co 3D, z~ograniczeniami na odpowiednich osiach
    \item Sprite'y w Unreal są renderowane jako płaskie meshe w przestrzeni 3D
\end{itemize}

\subsubsection{System Blueprintów vs C++}

Unreal oferuje dwa podejścia do programowania logiki gry:

\paragraph{Blueprinty} -- wizualny system skryptowy, który pozwala na szybkie prototypowanie bez pisania kodu. Dla prostych mechanik bullet-hell Blueprinty okazały się wystarczające i intuicyjne.

\paragraph{C++} -- dla bardziej wydajnościowo krytycznych elementów (jak system object poolingu) zalecane jest użycie C++. Jednak próg wejścia jest znacznie wyższy niż w przypadku C\# w Unity.

\subsubsection{Object Pooling w Unreal}

Implementacja object poolingu w~Unreal wymaga innego podejścia niż w~Unity. Zamiast prostego \texttt{SetActive(true/\allowbreak false)}, Unreal wykorzystuje:

\begin{itemize}
    \item \texttt{SetActorHiddenInGame()} -- kontrola widoczności
    \item \texttt{SetActorEnableCollision()} -- kontrola kolizji
    \item \texttt{SetActorTickEnabled()} -- kontrola aktualizacji logiki
\end{itemize}

Ta granularność daje większą kontrolę, ale wymaga więcej kodu do osiągnięcia tego samego efektu.

\subsubsection{Wyzwania napotkane w Unreal}

\begin{enumerate}
    \item \textbf{Brak natywnego 2D} -- konieczność ``symulowania'' środowiska 2D w silniku 3D
    \item \textbf{Czas kompilacji} -- kompilacja projektów C++ jest znacznie wolniejsza niż kompilacja C\# w Unity
    \item \textbf{Rozmiar projektu} -- nawet prosty projekt Unreal zajmuje wielokrotnie więcej miejsca na dysku
    \item \textbf{Dokumentacja} -- dla mniej popularnych zastosowań (jak gry 2D) dokumentacja jest ograniczona
    \item \textbf{Blueprinty i kontrola wersji} -- pliki Blueprintów są binarne, co utrudnia merge'owanie i code review
\end{enumerate}

\subsubsection{Pozytywne aspekty Unreal}

\begin{itemize}
    \item Potężny system materiałów i efektów wizualnych
    \item Wbudowane zaawansowane narzędzia profilowania
    \item Blueprinty umożliwiają szybkie prototypowanie przez osoby nietechniczne
    \item Doskonałe wsparcie dla grafiki 3D i fotorealizmu
\end{itemize}

\subsection{Porównanie doświadczeń implementacyjnych}

\begin{table}[htbp]
\centering
\caption{Porównanie doświadczeń z implementacji gry bullet-hell}
\label{tab:impl-comparison}
\begin{tabular}{|l|c|c|}
\hline
\textbf{Aspekt} & \textbf{Unity} & \textbf{Unreal Engine} \\
\hline
Czas instalacji (Linux) & $\sim$30 min & $\sim$2-4 h \\
\hline
Wsparcie natywne 2D & Tak & Nie (symulowane) \\
\hline
Język programowania & C\# & C++ / Blueprinty \\
\hline
Próg wejścia & Niski & Średni/Wysoki \\
\hline
Czas kompilacji & Szybki & Wolny (C++) \\
\hline
Object pooling & Prosty & Bardziej złożony \\
\hline
Hot reload & Tak & Ograniczony \\
\hline
Rozmiar projektu & Mały & Duży \\
\hline
\end{tabular}
\end{table}

\subsection{Wnioski z implementacji}

Doświadczenia z implementacji gry bullet-hell potwierdzają, że wybór silnika powinien być uzależniony od typu projektu:

\begin{enumerate}
    \item \textbf{Dla gier 2D} -- Unity oferuje znacznie lepsze wsparcie natywne, niższy próg wejścia i szybszy cykl iteracji
    \item \textbf{Dla gier 3D AAA} -- Unreal Engine dysponuje lepszymi narzędziami do tworzenia fotorealistycznej grafiki
    \item \textbf{Dla prototypowania} -- Unity pozwala na szybsze testowanie koncepcji dzięki hot reloadowi i prostszej konfiguracji
    \item \textbf{Dla zespołów mieszanych} -- Blueprinty Unreal mogą być wartościowe dla współpracy z designerami, choć problemy z kontrolą wersji stanowią wyzwanie
\end{enumerate}

Implementacja gry bullet-hell w~Unity zajęła około 60\% czasu potrzebnego na implementację analogicznej funkcjonalności w~Unreal Engine, głównie ze względu na natywne wsparcie 2D i~prostszy system object poolingu.
           % Doświadczenia z implementacji gry testowej
\clearpage
\section{Narzędzia profilowania wydajności}
\label{sec:narzedzia-profilowania}

Obiektywne porównanie wydajności silników gier wymaga zastosowania odpowiednich narzędzi pomiarowych. W~niniejszym rozdziale przedstawiono analizę dostępnych rozwiązań oraz uzasadnienie wyboru NVIDIA Nsight jako głównego narzędzia profilowania.

\subsection{Wbudowane narzędzia diagnostyczne silników}
\label{subsec:wbudowane-narzedzia}

Zarówno Unity, jak i~Unreal Engine oferują własne, wbudowane narzędzia do analizy wydajności. Każde z~nich posiada unikalne cechy dostosowane do specyfiki danego silnika.

\subsubsection{Unity Profiler}

Unity dostarcza rozbudowany profiler dostępny bezpośrednio w~edytorze (Window $\rightarrow$ Analysis $\rightarrow$ Profiler). Narzędzie to oferuje:

\begin{itemize}
    \item \textbf{CPU Profiler} -- analiza czasu wykonania poszczególnych funkcji, z~podziałem na kategorie (rendering, skrypty, fizyka, animacje)
    \item \textbf{GPU Profiler} -- pomiar czasu renderowania na karcie graficznej
    \item \textbf{Memory Profiler} -- szczegółowa analiza alokacji pamięci, wykrywanie wycieków
    \item \textbf{Audio Profiler} -- monitorowanie obciążenia systemu dźwiękowego
    \item \textbf{Physics Profiler} -- analiza wydajności silnika fizyki
    \item \textbf{Frame Debugger} -- krokowa analiza procesu renderowania pojedynczej klatki
\end{itemize}

Unity Profiler umożliwia również zdalne profilowanie aplikacji uruchomionej na urządzeniu docelowym (np.~smartfonie), co jest szczególnie przydatne przy optymalizacji gier mobilnych.

\subsubsection{Unreal Insights}

Unreal Engine oferuje narzędzie Unreal Insights, które zastąpiło starszy system Session Frontend. Kluczowe funkcjonalności obejmują:

\begin{itemize}
    \item \textbf{Timing Insights} -- precyzyjny pomiar czasu wykonania poszczególnych systemów silnika
    \item \textbf{Asset Loading Insights} -- analiza czasu ładowania zasobów
    \item \textbf{Memory Insights} -- monitorowanie alokacji i~dealokacji pamięci
    \item \textbf{Animation Insights} -- profilowanie systemu animacji
    \item \textbf{Network Insights} -- analiza ruchu sieciowego w~grach multiplayer
\end{itemize}

Dodatkowo Unreal Engine udostępnia komendy konsolowe (np.~\texttt{stat fps}, \texttt{stat unit}, \texttt{stat gpu}) pozwalające na szybki podgląd podstawowych metryk wydajności podczas rozgrywki.

\subsubsection{Ograniczenia narzędzi wbudowanych}

Pomimo rozbudowanych możliwości, wbudowane profilery silników posiadają istotne ograniczenia w~kontekście porównawczych badań wydajnościowych:

\begin{enumerate}
    \item \textbf{Brak standaryzacji metryk} -- każdy silnik definiuje i~mierzy parametry w~odmienny sposób, co utrudnia bezpośrednie porównania
    \item \textbf{Różna granularność danych} -- poziom szczegółowości raportów różni się między silnikami
    \item \textbf{Narzut profilowania} -- wbudowane profilery same generują obciążenie, które może być różne dla każdego silnika
    \item \textbf{Brak dostępu do danych niskopoziomowych} -- profilery silnikowe operują na poziomie abstrakcji silnika, nie hardware'u
    \item \textbf{Nieporównywalność formatów wyjściowych} -- dane eksportowane przez różne profilery mają odmienne struktury
\end{enumerate}

Z~powyższych powodów zdecydowano się na zastosowanie zewnętrznego, niezależnego od silnika narzędzia profilowania.

\subsection{NVIDIA Nsight Graphics}
\label{subsec:nvidia-nsight}

NVIDIA Nsight Graphics to profesjonalne narzędzie do profilowania i~debugowania aplikacji graficznych, oferujące głęboki wgląd w~działanie GPU niezależnie od używanego silnika czy API graficznego.

\subsubsection{Uzasadnienie wyboru}

Wybór NVIDIA Nsight jako głównego narzędzia pomiarowego podyktowany był następującymi czynnikami:

\begin{itemize}
    \item \textbf{Niezależność od silnika} -- Nsight analizuje aplikację na poziomie wywołań API graficznego (DirectX, Vulkan, OpenGL), co zapewnia porównywalność wyników między Unity a~Unreal Engine
    \item \textbf{Standaryzowane metryki} -- narzędzie dostarcza zunifikowany zestaw metryk sprzętowych (GPU utilization, memory bandwidth, shader throughput)
    \item \textbf{Minimalny narzut} -- profilowanie na poziomie sterownika generuje mniejsze zakłócenia niż profilery działające wewnątrz silnika
    \item \textbf{Dostęp do danych niskopoziomowych} -- możliwość analizy poszczególnych wywołań draw call, shaderów, transferów pamięci
    \item \textbf{Spójny format danych} -- wyniki z~obu silników mają identyczną strukturę, co ułatwia automatyzację analizy
\end{itemize}

\subsubsection{Możliwości narzędzia}

NVIDIA Nsight Graphics oferuje szereg funkcjonalności istotnych dla badań wydajnościowych:

\paragraph{Frame Profiler}
Główny moduł analizy wydajności, umożliwiający:
\begin{itemize}
    \item Przechwycenie i~analizę pojedynczej klatki (frame capture)
    \item Hierarchiczny widok wszystkich wywołań GPU
    \item Pomiar czasu wykonania każdego etapu renderowania
    \item Identyfikację wąskich gardeł (bottlenecks)
    \item Analizę wykorzystania jednostek obliczeniowych GPU
\end{itemize}

\paragraph{GPU Trace}
Moduł do długoterminowej analizy wydajności:
\begin{itemize}
    \item Rejestrowanie metryk przez określony czas (nie tylko pojedyncza klatka)
    \item Wykrywanie spadków wydajności i~ich przyczyn
    \item Analiza zmienności czasów klatek (frame time variance)
    \item Korelacja obciążenia GPU z~wydarzeniami w~grze
\end{itemize}

\paragraph{Shader Profiler}
Narzędzie do optymalizacji shaderów:
\begin{itemize}
    \item Analiza wydajności poszczególnych shaderów
    \item Identyfikacja nieefektywnych instrukcji
    \item Pomiar occupancy (wykorzystania jednostek obliczeniowych)
    \item Sugestie optymalizacyjne
\end{itemize}

\subsubsection{Konfiguracja środowiska pomiarowego}

Przed przeprowadzeniem pomiarów skonfigurowano środowisko w~następujący sposób:

\begin{enumerate}
    \item Wyłączenie V-Sync w~obu silnikach (eliminacja sztucznego ograniczenia FPS)
    \item Ustawienie identycznej rozdzielczości renderowania (1920$\times$1080)
    \item Wyłączenie dynamicznego skalowania rozdzielczości
    \item Ustawienie stałej częstotliwości zegara GPU (eliminacja power throttlingu)
    \item Zamknięcie zbędnych procesów w~tle
    \item Oczekiwanie na ustabilizowanie temperatury GPU przed pomiarem
\end{enumerate}

\subsection{Przetwarzanie danych z~Nsight}
\label{subsec:przetwarzanie-nsight}

Dane zebrane przez NVIDIA Nsight wymagają odpowiedniego przetworzenia w~celu uzyskania porównywalnych metryk.

\subsubsection{Eksport danych}

Nsight umożliwia eksport danych w~kilku formatach:
\begin{itemize}
    \item \textbf{CSV} -- tabularyczne dane liczbowe, idealne do dalszej analizy
    \item \textbf{JSON} -- strukturalne dane z~pełną hierarchią wywołań
    \item \textbf{HTML Report} -- czytelny raport z~wykresami (mniej przydatny do automatyzacji)
\end{itemize}

W~niniejszej pracy wykorzystano format CSV ze względu na łatwość importu do narzędzi analizy statystycznej.

\subsubsection{Kluczowe metryki}

Z~danych eksportowanych przez Nsight wyodrębniono następujące metryki:

\begin{table}[htbp]
\centering
\caption{Kluczowe metryki wydajnościowe z~NVIDIA Nsight}
\label{tab:metryki-nsight}
\begin{tabular}{|>{\raggedright\arraybackslash}p{4cm}|>{\raggedright\arraybackslash}p{3cm}|>{\raggedright\arraybackslash}p{5.5cm}|}
\hline
\textbf{Metryka} & \textbf{Jednostka} & \textbf{Opis} \\
\hline
Frame Time & ms & Całkowity czas renderowania klatki \\
\hline
GPU Duration & ms & Czas pracy GPU (bez CPU overhead) \\
\hline
Draw Calls & liczba & Ilość wywołań rysowania na klatkę \\
\hline
Triangles Rendered & liczba & Liczba wyrenderowanych trójkątów \\
\hline
GPU Memory Used & MB & Zużycie pamięci VRAM \\
\hline
SM Occupancy & \% & Wykorzystanie jednostek obliczeniowych \\
\hline
Memory Bandwidth & GB/s & Przepustowość pamięci GPU \\
\hline
\end{tabular}
\end{table}

\subsubsection{Metodyka pomiarów}

Dla każdej konfiguracji testowej przeprowadzono serię pomiarów według następującego protokołu:

\begin{enumerate}
    \item Uruchomienie aplikacji i~oczekiwanie 30 sekund na stabilizację
    \item Rozpoczęcie rejestracji GPU Trace (czas trwania: 60 sekund)
    \item Przechwycenie 10 pojedynczych klatek w~równych odstępach czasu
    \item Zakończenie rejestracji i~eksport danych
    \item Powtórzenie procedury 3 razy dla każdej konfiguracji
\end{enumerate}

Wyniki uśredniono, odrzucając wartości odstające (outliers) zidentyfikowane metodą IQR (InterQuartile Range).

\subsubsection{Automatyzacja analizy}

W~celu zapewnienia powtarzalności i~eliminacji błędów ludzkich, proces analizy danych został częściowo zautomatyzowany za pomocą skryptów Python. Główne etapy obejmowały:

\begin{itemize}
    \item Parsowanie plików CSV eksportowanych z~Nsight
    \item Agregację danych z~wielu sesji pomiarowych
    \item Obliczanie statystyk opisowych (średnia, mediana, odchylenie standardowe)
    \item Generowanie wykresów porównawczych
    \item Eksport wyników do formatu LaTeX (tabele)
\end{itemize}

\subsection{Podsumowanie wyboru narzędzi}
\label{subsec:podsumowanie-narzedzi}

Zastosowanie NVIDIA Nsight jako głównego narzędzia profilowania zapewnia:

\begin{enumerate}
    \item \textbf{Obiektywność} -- pomiary wykonywane na tym samym poziomie abstrakcji dla obu silników
    \item \textbf{Porównywalność} -- identyczne metryki i~format danych
    \item \textbf{Wiarygodność} -- niskopoziomowe pomiary eliminują artefakty wprowadzane przez profilery silnikowe
    \item \textbf{Powtarzalność} -- standaryzowana procedura pomiarowa
\end{enumerate}

Wbudowane profilery Unity i~Unreal Engine pozostają cennym narzędziem podczas procesu optymalizacji, jednak do celów badawczych wymagających bezpośredniego porównania między silnikami, zewnętrzne narzędzie oferuje znaczące przewagi metodologiczne.
      % Narzędzia profilowania wydajności
\clearpage
\section{Testy wydajności}
\label{sec:testy-wydajnosci}

\subsection{Metodyka przeprowadzania testów}
\subsubsection{Przygotowanie środowiska testowego}
% Opis procesu przygotowania spójnego środowiska dla wszystkich silników

\subsubsection{Standaryzacja warunków testowych}
% Zapewnienie porównywalności wyników między różnymi silnikami

\subsection{Test renderowania 2D}
\subsubsection{Założenia testu}
% Opis scenariusza testowego dla grafiki 2D

\subsubsection{Wyniki pomiarów}
% Tabele i wykresy przedstawiające wyniki dla każdego silnika

\subsubsection{Analiza wyników}
% Interpretacja uzyskanych danych

\subsection{Test renderowania 3D}
\subsubsection{Scenariusz podstawowy}
% Test renderowania prostej sceny 3D

\subsubsection{Scenariusz zaawansowany}
% Test złożonych scen z wieloma obiektami, oświetleniem, cieniami

\subsubsection{Porównanie wyników}
% Analiza porównawcza wydajności renderowania 3D

\subsection{Test systemów fizyki}
\subsubsection{Symulacja kolizji}
% Testy wydajności przy różnej liczbie obiektów fizycznych

\subsubsection{Symulacja płynów i cząstek}
% Testy zaawansowanych systemów fizyki

\subsection{Test zużycia zasobów systemowych}
\subsubsection{Zużycie pamięci RAM}
% Pomiary zużycia pamięci operacyjnej

\subsubsection{Obciążenie procesora}
% Analiza wykorzystania CPU

\subsubsection{Zużycie pamięci GPU}
% Pomiary zużycia pamięci karty graficznej

\subsection{Test wydajności na różnych platformach}
\subsubsection{Testy na PC (Windows/Linux)}
% Porównanie wydajności na systemach desktop

\subsubsection{Testy na urządzeniach mobilnych}
% Analiza wydajności na platformach mobilnych (jeśli dotyczy)

\subsection{Podsumowanie wyników testów wydajności}
% Zestawienie wszystkich uzyskanych wyników w formie tabelarycznej i graficznej
          % Testy wydajności
\clearpage
\section{Analiza możliwości i funkcjonalności}

\subsection{Metodyka oceny funkcjonalności}
\subsubsection{Kryteria oceny}
% Opis systemu punktowego lub skali ocen

\subsubsection{Proces walidacji}
% Sposób weryfikacji dostępności i jakości funkcji

\subsection{Analiza możliwości renderingu}
\subsubsection{Wsparcie dla różnych technik renderingu}
% Forward rendering, deferred rendering, ray tracing, etc.

\subsubsection{Systemy materiałów i shaderów}
% Możliwości tworzenia i edycji materiałów

\subsubsection{Systemy oświetlenia}
% Real-time lighting, baked lighting, global illumination

\subsection{Systemy fizyki i symulacji}
\subsubsection{Rigid body physics}
% Analiza możliwości symulacji ciał sztywnych

\subsubsection{Soft body physics}
% Wsparcie dla symulacji ciał miękkich

\subsubsection{Systemy cząstek}
% Możliwości tworzenia efektów cząsteczkowych

\subsection{Systemy audio}
\subsubsection{Wsparcie formatów audio}
% Obsługiwane formaty i kodeki

\subsubsection{Przestrzenny dźwięk 3D}
% Możliwości pozycjonowania dźwięku w przestrzeni

\subsubsection{Efekty audio i DSP}
% Dostępne efekty i systemy przetwarzania dźwięku

\subsection{Narzędzia deweloperskie}
\subsubsection{Edytory wizualne}
% Analiza jakości i funkcjonalności edytorów

\subsubsection{Systemy debugowania}
% Dostępne narzędzia do debugowania

\subsubsection{Profilowanie wydajności}
% Wbudowane narzędzia do analizy wydajności

\subsection{Wsparcie dla platform docelowych}
\subsubsection{Platformy desktop}
% Windows, macOS, Linux

\subsubsection{Platformy mobilne}
% iOS, Android

\subsubsection{Konsole}
% PlayStation, Xbox, Nintendo Switch

\subsubsection{Platformy VR/AR}
% Wsparcie dla rzeczywistości wirtualnej i rozszerzonej

\subsection{Ekosystem i rozszerzalność}
\subsubsection{Asset Store / Marketplace}
% Dostępność gotowych zasobów i rozszerzeń

\subsubsection{Wsparcie społeczności}
% Aktywność społeczności, dostępność pomocy

\subsubsection{Dokumentacja i materiały edukacyjne}
% Jakość oficjalnej dokumentacji i tutoriali
        % Analiza możliwości i funkcjonalności
\clearpage
\section{Porównanie wyników i analiza}

\subsection{Synteza wyników badań}
\subsubsection{Zestawienie wyników testów wydajności}
% Kompleksowe podsumowanie wszystkich pomiarów wydajności

\subsubsection{Zestawienie analizy funkcjonalności}
% Podsumowanie oceny możliwości każdego silnika

\subsection{Analiza wielokryterialna}
\subsubsection{Macierz porównawcza}
% Tabela porównująca wszystkie silniki według wszystkich kryteriów

\subsubsection{Analiza wag kryteriów}
% Określenie ważności poszczególnych aspektów dla różnych typów projektów

\subsection{Przypadki użycia}
\subsubsection{Gry indie}
% Rekomendacje dla małych zespołów deweloperskich

\subsubsection{Gry mobilne}
% Najlepsze rozwiązania dla platform mobilnych

\subsubsection{Gry AAA}
% Analiza przydatności dla dużych projektów komercyjnych

\subsubsection{Gry VR/AR}
% Specyficzne wymagania dla rzeczywistości wirtualnej

\subsection{Analiza korelacji}
\subsubsection{Związek między wydajnością a funkcjonalnością}
% Analiza statystyczna zależności między różnymi parametrami

\subsubsection{Wpływ złożoności na użyteczność}
% Badanie relacji między liczbą funkcji a łatwością użycia

\subsection{Ograniczenia badań}
\subsubsection{Ograniczenia metodologiczne}
% Omówienie ograniczeń zastosowanej metodologii

\subsubsection{Ograniczenia techniczne}
% Wpływ środowiska testowego na wyniki

\subsubsection{Ograniczenia czasowe}
% Wpływ dynamicznego rozwoju silników na aktualność wyników

\subsection{Weryfikacja hipotez badawczych}
% Odniesienie uzyskanych wyników do postawionych na początku hipotez

\subsection{Implikacje praktyczne}
% Wnioski dla praktyki tworzenia gier i wyboru silnika
        % Porównanie wyników i analiza
\clearpage
\section{Podsumowanie i wnioski}

\subsection{Główne wyniki badań}
\subsubsection{Odpowiedzi na pytania badawcze}
% Zwięzłe odpowiedzi na postawione na początku pytania

\subsubsection{Weryfikacja hipotez}
% Potwierdzenie lub odrzucenie hipotez badawczych

\subsection{Wnioski praktyczne}
\subsubsection{Rekomendacje dla deweloperów}
% Praktyczne wskazówki dotyczące wyboru silnika gier

\subsubsection{Wytyczne dla różnych typów projektów}
% Szczegółowe rekomendacje w zależności od rodzaju gry

\subsection{Wkład naukowy}
\subsubsection{Nowatorskie aspekty badań}
% Elementy które wnoszą nową wiedzę do dziedziny

\subsubsection{Znaczenie dla branży}
% Potencjalny wpływ wyników na przemysł gier

\subsection{Ograniczenia i przyszłe badania}
\subsubsection{Identyfikacja ograniczeń}
% Szczere omówienie ograniczeń przeprowadzonych badań

\subsubsection{Propozycje dalszych badań}
% Sugestie dla przyszłych prac badawczych w tej dziedzinie

\subsubsection{Rozwój metodologii}
% Możliwości udoskonalenia zastosowanej metodologii

\subsection{Refleksje końcowe}
% Osobiste refleksje autora na temat przeprowadzonych badań i ich znaczenia

\subsection{Znaczenie wyników w kontekście rozwoju technologii}
% Umiejscowienie wyników w szerszym kontekście rozwoju technologii gier

% Ten rozdział powinien stanowić logiczne zamknięcie całej pracy
% i pokazać, że cele zostały osiągnięte
              % Podsumowanie i wnioski

%---------------
% Bibliografia
%---------------
\cleardoublepage % Zaczynamy od nieparzystej strony
\begingroup
\emergencystretch=1em
\printbibliography
\endgroup
\clearpage

% Wykaz symboli i skrótów.
% Pamiętaj, żeby posortować symbole alfabetycznie
% we własnym zakresie. Makro \acronymlist
% generuje właściwy tytuł sekcji, w zależności od języka.
% Makro \acronym dodaje skrót/symbol do listy,
% zapewniając podstawowe formatowanie.
\acronymlist
\acronym{EiTI}{Wydział Elektroniki i Technik Informacyjnych}
\acronym{PW}{Politechnika Warszawska}
\acronym{WEIRD}{ang. \emph{Western, Educated, Industrialized, Rich and Democratic}}
\vspace{0.8cm}

%--------------------------------------
% Spisy: rysunków, tabel, załączników
%--------------------------------------
\pagestyle{plain}

\listoffigurestoc    % Spis rysunków.
\vspace{1cm}         % vertical space
\listoftablestoc     % Spis tabel.
\vspace{1cm}         % vertical space
\listofappendicestoc % Spis załączników

%-------------
% Załączniki
%-------------

% Obrazki i tabele w załącznikach nie trafiają do spisów
\captionsetup[figure]{list=no}
\captionsetup[table]{list=no}

% Załącznik 1
\clearpage
\appendix{Nazwa załącznika 1}
\lipsum[1-3]
\begin{figure}[!h]
	\centering \includegraphics[width=0.5\linewidth]{logopw2.png}
	\caption{Obrazek w załączniku.}
\end{figure}
\lipsum[4-7]

% Załącznik 2
\clearpage
\appendix{Nazwa załącznika 2}
\lipsum[1-2]
\begin{table}[!h] \centering
    \caption{Tabela w załączniku.}
    \begin{tabular} {| c | c | r |} \hline
        Kolumna 1       & Kolumna 2 & Liczba \\ \hline\hline
        cell1           & cell2     & 60     \\ \hline
        \multicolumn{2}{|r|}{Suma:} & 123,45 \\ \hline
    \end{tabular}
\end{table}
\lipsum[3-4]

% Używając powyższych spisów jako szablonu,
% możesz dodać również swój własny wykaz,
% np. spis algorytmów.

\end{document} % Dobranoc.
