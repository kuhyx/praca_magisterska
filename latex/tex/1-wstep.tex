\clearpage % Rozdziały zaczynamy od nowej strony.
\section{Wstęp}

\subsection{Motywacja i cel pracy}
Współczesny rynek gier komputerowych charakteryzuje się dynamicznym rozwojem technologicznym i~rosnącymi wymaganiami zarówno twórców, jak i~graczy. Wybór odpowiedniego silnika gier jest kluczową decyzją, która wpływa na cały proces tworzenia gry, jej wydajność oraz możliwości techniczne.

Celem niniejszej pracy jest kompleksowe porównanie wydajności i~możliwości współczesnych silników gier komputerowych, ze szczególnym uwzględnieniem ich wpływu na proces tworzenia gier oraz końcową jakość produktu.

\subsection{Zakres pracy}
Praca obejmuje analizę następujących aspektów:
\begin{itemize}
    \item Wydajność renderowania grafiki 2D i~3D
    \item Możliwości i~funkcjonalności oferowane przez różne silniki
    \item Łatwość użycia i~krzywa uczenia się
    \item Wsparcie dla różnych platform docelowych
    \item Ekosystem narzędzi i~społeczność deweloperska
\end{itemize}

\subsection{Wybór gry testowej -- gatunek bullet hell}
\label{subsec:bullet-hell}

W~celu przeprowadzenia praktycznych testów wydajnościowych zdecydowano się na implementację gry z~gatunku \textbf{bullet hell} (dosł.~,,piekło pocisków''), znanego również jako \textbf{danmaku} (z~jap.~,,kurtyna pocisków'') lub \textbf{manic shooter}.

\subsubsection{Charakterystyka gatunku}

Bullet hell to podgatunek gier typu shoot 'em up (strzelanka), w~którym gracz steruje zwykle niewielkim statkiem kosmicznym lub postacią, mierząc się z~falami przeciwników wystrzeliwujących ogromne ilości pocisków tworzących skomplikowane wzory na ekranie. Kluczowe cechy gatunku obejmują:

\begin{itemize}
    \item \textbf{Masowa ilość pocisków} -- na ekranie jednocześnie może znajdować się od kilkuset do kilku tysięcy pocisków, tworzących złożone formacje geometryczne
    \item \textbf{Precyzyjne hitboxy} -- obszar kolizji postaci gracza jest znacznie mniejszy niż jej wizualna reprezentacja (często ograniczony do kilku pikseli), co umożliwia nawigację między gęstymi wzorami pocisków
    \item \textbf{Wzory pocisków} -- przeciwnicy wystrzeliwują pociski według określonych algorytmów, tworząc spirale, fale, rozgałęzienia i~inne formacje
    \item \textbf{Ciągły ruch} -- gracz musi nieustannie przemieszczać się po ekranie, unikając kolizji
    \item \textbf{Eskalacja trudności} -- wraz z~postępem gry wzrasta liczba przeciwników i~gęstość pocisków
\end{itemize}

Klasyczne przykłady gatunku to serie \textit{Touhou Project}, \textit{DoDonPachi}, \textit{Ikaruga} oraz \textit{Geometry Wars}.

\subsubsection{Uzasadnienie wyboru gatunku}

Gatunek bullet hell został wybrany jako podstawa testów wydajnościowych z~następujących powodów:

\begin{enumerate}
    \item \textbf{Intensywne wykorzystanie zasobów} -- jednoczesne renderowanie setek lub tysięcy obiektów (pocisków) stanowi znaczące obciążenie dla systemu renderowania
    
    \item \textbf{Testowanie zarządzania pamięcią} -- ciągłe tworzenie i~niszczenie obiektów pocisków eksponuje różnice w~implementacji garbage collectora (Unity/C\#) versus ręcznego zarządzania pamięcią (Unreal/C++)
    
    \item \textbf{Wymagania systemu fizyki} -- wykrywanie kolizji między graczem a~setkami pocisków w~każdej klatce obciąża system fizyki
    
    \item \textbf{Prostota implementacji} -- podstawowa mechanika gry jest stosunkowo prosta koncepcyjnie, co pozwala skupić się na porównaniu wydajności, a~nie złożoności logiki gry
    
    \item \textbf{Skalowalność testu} -- łatwo kontrolować poziom obciążenia poprzez modyfikację liczby aktywnych pocisków i~przeciwników
    
    \item \textbf{Reprezentatywność dla gier 2D} -- gatunek jest typowym przedstawicielem gier 2D, co pozwala ocenić wsparcie silników dla tego segmentu rynku
    
    \item \textbf{Wymuszenie optymalizacji} -- ze względu na ekstremalną liczbę obiektów, implementacja bullet hell wymusza stosowanie technik optymalizacyjnych (object pooling, spatial partitioning), których efektywność może różnić się między silnikami
\end{enumerate}

\subsubsection{Parametry gry testowej}

Zaimplementowana gra testowa charakteryzuje się następującymi parametrami:

\begin{itemize}
    \item Czas rozgrywki: 90 sekund (tryb przetrwania)
    \item Eskalacja trudności: liniowy wzrost częstotliwości spawnu przeciwników
    \item Typy przeciwników: 3 warianty z~różnymi wzorami strzelania
    \item Maksymalna liczba jednoczesnych pocisków: do 500 obiektów
    \item System punktacji oparty na eliminacji przeciwników
    \item Object pooling dla pocisków (eliminacja alokacji w~runtime)
\end{itemize}

Te parametry zapewniają wystarczające obciążenie systemu do ujawnienia różnic wydajnościowych między silnikami, pozostając jednocześnie w~granicach typowych dla gier indie z~tego gatunku.

\subsection{Struktura pracy}

Praca składa się z~następujących rozdziałów:

\begin{enumerate}
    \item \textbf{Wstęp} -- wprowadzenie do tematyki, motywacja, cel i~zakres pracy
    \item \textbf{Przegląd literatury} -- analiza istniejących badań porównawczych silników gier
    \item \textbf{Charakterystyka silników} -- szczegółowy opis Unity i~Unreal Engine
    \item \textbf{Metodologia} -- opis metodyki badawczej i~kryteriów porównania
    \item \textbf{Analiza wywiadów} -- wyniki badań jakościowych z~deweloperami
    \item \textbf{Implementacja gry testowej} -- doświadczenia z~tworzenia gry w~obu silnikach
    \item \textbf{Narzędzia profilowania} -- opis NVIDIA Nsight i~metodyki pomiarów
    \item \textbf{Testy wydajności} -- wyniki pomiarów wydajnościowych
    \item \textbf{Analiza możliwości} -- porównanie funkcjonalności silników
    \item \textbf{Porównanie wyników} -- synteza i~analiza zebranych danych
    \item \textbf{Podsumowanie} -- wnioski i~rekomendacje
\end{enumerate}

\subsection{Metodologia}

W~pracy zastosowano metodologię badawczą łączącą podejście ilościowe z~jakościowym:

\begin{itemize}
    \item \textbf{Testy wydajnościowe} -- obiektywne pomiary z~wykorzystaniem NVIDIA Nsight Graphics, zapewniające porównywalność wyników między silnikami
    \item \textbf{Wywiady z~deweloperami} -- badania jakościowe dostarczające kontekstu praktycznego użytkowania silników
    \item \textbf{Implementacja porównawcza} -- stworzenie identycznej gry w~obu silnikach, dokumentując różnice w~procesie deweloperskim
    \item \textbf{Analiza dokumentacji} -- przegląd oficjalnej dokumentacji i~materiałów edukacyjnych
\end{itemize}

Takie wieloaspektowe podejście pozwala na kompleksową ocenę silników, uwzględniającą zarówno mierzalne parametry techniczne, jak i~subiektywne doświadczenia użytkowników.
