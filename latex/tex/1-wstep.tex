\clearpage % Rozdziały zaczynamy od nowej strony.
\section{Wstęp}

\subsection{Motywacja i cel pracy}
Współczesny rynek gier komputerowych charakteryzuje się dynamicznym rozwojem technologicznym i rosnącymi wymaganiami zarówno twórców, jak i graczy. Wybór odpowiedniego silnika gier jest kluczową decyzją, która wpływa na cały proces tworzenia gry, jej wydajność oraz możliwości techniczne.

Celem niniejszej pracy jest kompleksowe porównanie wydajności i możliwości współczesnych silników gier komputerowych, ze szczególnym uwzględnieniem ich wpływu na proces tworzenia gier oraz końcową jakość produktu.

\subsection{Zakres pracy}
Praca obejmuje analizę następujących aspektów:
\begin{itemize}
    \item Wydajność renderowania grafiki 2D i 3D
    \item Możliwości i funkcjonalności oferowane przez różne silniki
    \item Łatwość użycia i krzywa uczenia się
    \item Wsparcie dla różnych platform docelowych
    \item Ekosystem narzędzi i społeczność deweloperska
\end{itemize}

\subsection{Struktura pracy}
Praca składa się z następujących rozdziałów:
% TODO: Dodać opis struktury po ukończeniu wszystkich rozdziałów

\subsection{Metodologia}
W pracy zastosowano metodologię badawczą opartą na analizie porównawczej, testach wydajnościowych oraz studium przypadków rzeczywistych projektów implementowanych w różnych silnikach gier.

% \ref wyrenderuje się jako 'Reference to image 1.1'
\lipsum[2] Reference to image \ref{fig:tradycyjne-logo-pw}.

% Lista punktowana
% Parametr label ustawia symbol punktora
\begin{itemize}
    \item Item 1:
    \begin{itemize}[label=---]
        \item item 1.1;
        \item item 1.2;
    \end{itemize}
    \item Item 2;
    \item Item 3.
\end{itemize}

\lipsum[3]

% Lista numerowana w formacie 1.a).ii
% Tutaj również można stosować \label
\begin{enumerate}
    \item Item 1:
    \begin{enumerate}
        \item item 1.1;
        \item item 1.2:
        \begin{enumerate}
            \item item 1.2.1;
            \item item 1.2.2;
        \end{enumerate}
        \item item 1.3;
    \end{enumerate}
    \item Item 2;
    \item Item 3.
\end{enumerate}

% Przypis dolny \footnote
\lipsum[4] Lorem ipsum dolor sit amet\footnote{Lorem ipsum dolor sit amet, consectetur adipiscing elit, sed do eiusmod tempor incididunt ut labore et dolore magna aliqua. Ut enim ad minim veniam, quis nostrud exercitation ullamco laboris nisi ut aliquip ex ea commodo consequat.}, consectetur adipiscing elit.

% Przykładowa tabela: wyśrodkowana i renderowana
% w miejscu wstawienia: !h = !h[ere]
% Domyślnie tabele trafiają na górę strony
\begin{table}[!h] \centering
    % Podpis tabeli umieszczamy od góry
    \caption{Przykładowa tabela.}
    \label{tab:tabela1}

    % Tabela z trzema kolumnami:
    % dwie wyrównanie do środka [c], a ostatnia do prawej [r]
    % szerokość kolumn automatyczna (równa szerokości tekstu)
    \begin{tabular}{| c | c | r |} \hline
        Kolumna 1       & Kolumna 2 & Liczba \\ \hline\hline
        cell1           & cell2     & 60     \\ \hline
        cell4           & cell5     & 43     \\ \hline
        cell7           & cell8     & 20,45  \\ \hline
        % Komórka o szerokości dwóch kolumn, wyrównana do prawej
        % Przypisy dolne w tabelach wstawiamy przez \tablefootnote
        \multicolumn{2}{|r|}{Suma\tablefootnote{Table footnote.}} & 123,45 \\ \hline
    \end{tabular}

\end{table}

Lorem ipsum dolor sit amet.
