\clearpage
\section{Charakterystyka współczesnych silników gier}

\subsection{Kryteria wyboru silników do analizy}
W ramach niniejszej pracy wybrano reprezentatywne silniki gier reprezentujące różne segmenty rynku:
\begin{itemize}
    \item Popularność i rozpowszechnienie
    \item Różnorodność technologiczna
    \item Wsparcie dla różnych platform
    \item Dostępność dokumentacji i narzędzi
\end{itemize}

\subsection{Unity}
\subsubsection{Architektura i technologie}
% Opis architektury silnika Unity, stosowanych technologii

\subsubsection{Możliwości i funkcjonalności}
% Szczegółowy opis oferowanych funkcji

\subsubsection{Narzędzia deweloperskie}
% Edytor, debugger, profiler, etc.

\subsection{Unreal Engine}
\subsubsection{Architektura i technologie}
% Opis architektury Unreal Engine

\subsubsection{Możliwości i funkcjonalności}
% Blueprint system, C++ support, rendering pipeline

\subsubsection{Narzędzia deweloperskie}
% Unreal Editor, Visual Scripting, etc.

\subsection{Godot}
\subsubsection{Architektura i technologie}
% Opis architektury silnika Godot

\subsubsection{Możliwości i funkcjonalności}
% GDScript, scene system, node-based architecture

\subsubsection{Narzędzia deweloperskie}
% Integrated editor, scripting tools

\subsection{Inne analizowane silniki}
% Krótka charakterystyka dodatkowych silników (np. CryEngine, GameMaker Studio)

\subsection{Porównanie tabelaryczne podstawowych cech}
% Tabela porównawcza kluczowych parametrów
