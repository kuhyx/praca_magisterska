\clearpage
\section{Charakterystyka współczesnych silników gier}

\subsection{Kryteria wyboru silników do analizy}

Rynek silników gier komputerowych oferuje szeroki wachlarz rozwiązań, od prostych frameworków 2D po zaawansowane środowiska do tworzenia fotorealistycznych produkcji AAA. W ramach niniejszej pracy zdecydowano się na dogłębną analizę dwóch silników: \textbf{Unity} oraz \textbf{Unreal Engine}. Wybór ten podyktowany był kilkoma kluczowymi czynnikami:

\begin{itemize}
    \item \textbf{Dominacja rynkowa} -- według danych z~2024 roku, Unity i~Unreal Engine wspólnie obsługują ponad 70\% globalnego rynku gier komputerowych i~mobilnych
    \item \textbf{Reprezentatywność podejść architektonicznych} -- silniki reprezentują odmienne filozofie: Unity opiera się na języku C\# z~garbage collectorem, a~Unreal wykorzystuje natywny C++ z~ręcznym zarządzaniem pamięcią
    \item \textbf{Różnorodność zastosowań} -- Unity tradycyjnie dominuje w segmencie gier mobilnych i indie, natomiast Unreal jest preferowany w produkcjach AAA i projektach wymagających fotorealistycznej grafiki
    \item \textbf{Dostępność} -- oba silniki oferują darmowe wersje dla małych zespołów i projektów edukacyjnych, co czyni je dostępnymi dla szerokiego grona deweloperów
    \item \textbf{Bogata dokumentacja} -- zarówno Unity jak i Unreal dysponują rozbudowaną dokumentacją oficjalną oraz aktywnymi społecznościami
\end{itemize}

\subsection{Unity}
\label{subsec:unity}

\subsubsection{Wprowadzenie i historia}

Unity to wieloplatformowy silnik gier stworzony przez Unity Technologies, którego pierwsza wersja została wydana w 2005 roku jako ekskluzywne narzędzie dla systemu macOS. Od tego czasu silnik przeszedł znaczącą ewolucję, stając się jednym z najpopularniejszych rozwiązań do tworzenia gier na świecie.

Kluczowym momentem w historii Unity było wprowadzenie w 2010 roku darmowej wersji silnika (Unity Free), co znacząco obniżyło barierę wejścia dla początkujących deweloperów i małych studiów. Decyzja ta przyczyniła się do eksplozji popularności silnika w segmencie gier mobilnych oraz indie.

Unity wykorzystuje język programowania \textbf{C\#} działający na platformie .NET/Mono, co zapewnia:
\begin{itemize}
    \item Automatyczne zarządzanie pamięcią poprzez garbage collector
    \item Bezpieczeństwo typów i obsługę wyjątków
    \item Bogatą bibliotekę standardową
    \item Stosunkowo łagodną krzywą uczenia dla programistów znających Javę lub podobne języki
\end{itemize}

Architektura Unity opiera się na wzorcu \textit{GameObject-Component}, gdzie każdy obiekt w scenie (GameObject) może posiadać dowolną liczbę komponentów definiujących jego zachowanie. Podejście to promuje kompozycję nad dziedziczeniem i ułatwia tworzenie modularnego kodu.

\subsubsection{Możliwości i funkcjonalności}

Unity oferuje kompleksowy zestaw narzędzi do tworzenia gier 2D i 3D:

\begin{itemize}
    \item \textbf{Rendering} -- wsparcie dla wielu pipeline'ów renderowania: Built-in, Universal Render Pipeline (URP) dla platform mobilnych oraz High Definition Render Pipeline (HDRP) dla wysokiej jakości grafiki
    \item \textbf{Fizyka} -- integracja z silnikami PhysX (3D) i Box2D (2D)
    \item \textbf{Animacja} -- system Mecanim z obsługą maszyn stanów i blendingu animacji
    \item \textbf{Audio} -- wbudowany system dźwięku przestrzennego
    \item \textbf{UI} -- dwa systemy interfejsu użytkownika: legacy uGUI oraz nowoczesny UI Toolkit
    \item \textbf{Multiplayer} -- Netcode for GameObjects oraz integracja z usługami sieciowymi
\end{itemize}

\subsubsection{Narzędzia deweloperskie}

Edytor Unity zapewnia intuicyjny interfejs graficzny z następującymi funkcjonalnościami:
\begin{itemize}
    \item Hierarchiczny widok sceny z możliwością edycji w czasie rzeczywistym
    \item Inspektor właściwości z obsługą serializacji pól poprzez atrybut \texttt{[SerializeField]}
    \item Wbudowany profiler wydajności (CPU, GPU, pamięć)
    \item Asset Store -- marketplace z gotowymi zasobami i rozszerzeniami
    \item Obsługa hot reload -- możliwość edycji kodu podczas działania gry
\end{itemize}

\subsection{Unreal Engine}
\label{subsec:unreal}

\subsubsection{Wprowadzenie i historia}

Unreal Engine to silnik gier stworzony przez Epic Games, którego historia sięga 1998 roku, kiedy to zadebiutował wraz z grą \textit{Unreal}. Od początku silnik był projektowany z myślą o tworzeniu gier pierwszoosobowych (FPS) o wysokiej jakości graficznej, co nadal pozostaje jego mocną stroną.

Przełomowym momentem było wydanie Unreal Engine 4 w 2014 roku na licencji royalty-free (5\% od przychodów powyżej \$1 miliona), a następnie Unreal Engine 5 w 2022 roku, wprowadzającego rewolucyjne technologie takie jak Nanite (wirtualizowana geometria) i Lumen (globalne oświetlenie w czasie rzeczywistym).

Unreal Engine wykorzystuje język programowania \textbf{C++} z rozszerzeniami specyficznymi dla silnika (makra UE), co zapewnia:
\begin{itemize}
    \item Maksymalną wydajność dzięki kompilacji do kodu natywnego
    \item Pełną kontrolę nad zarządzaniem pamięcią
    \item Dostęp do kodu źródłowego silnika (po uzyskaniu licencji)
    \item Strome krzywe uczenia, szczególnie dla programistów bez doświadczenia w C++
\end{itemize}

Dodatkowo Unreal oferuje system \textbf{Blueprints} -- wizualny język skryptowy pozwalający na tworzenie logiki gry bez pisania kodu. Blueprinty są szczególnie przydatne dla designerów i artystów, choć dla złożonych systemów mogą być mniej wydajne niż natywny C++.

\subsubsection{Możliwości i funkcjonalności}

Unreal Engine wyróżnia się zaawansowanymi możliwościami graficznymi:

\begin{itemize}
    \item \textbf{Rendering} -- fotorealistyczna grafika z obsługą ray tracingu, Nanite i Lumen
    \item \textbf{Fizyka} -- silnik Chaos Physics z obsługą destrukcji i symulacji ciał miękkich
    \item \textbf{Animacja} -- Control Rig, Animation Blueprints, IK Retargeting
    \item \textbf{Landscape} -- zaawansowane narzędzia do tworzenia dużych terenów
    \item \textbf{Niagara} -- system efektów cząsteczkowych nowej generacji
    \item \textbf{Sequencer} -- narzędzie do tworzenia cinematików i cutscen
\end{itemize}

\subsubsection{Narzędzia deweloperskie}

Unreal Editor oferuje rozbudowane środowisko deweloperskie:
\begin{itemize}
    \item Edytor poziomów z obsługą streamingu i Level of Detail (LOD)
    \item Blueprint Visual Scripting -- programowanie wizualne
    \item Material Editor -- węzłowy edytor materiałów
    \item Wbudowany profiler z analizą GPU/CPU i pamięci
    \item Marketplace -- sklep z zasobami i pluginami
    \item Live Coding -- eksperymentalne wsparcie dla hot reload w C++
\end{itemize}

\subsection{Porównanie architektoniczne}

\begin{table}[ht]
\centering
\caption{Porównanie kluczowych cech Unity i Unreal Engine}
\label{tab:unity-vs-unreal}
\begin{tabular}{|l|c|c|}
\hline
\textbf{Cecha} & \textbf{Unity} & \textbf{Unreal Engine} \\
\hline
Język programowania & C\# & C++ / Blueprints \\
\hline
Zarządzanie pamięcią & Automatyczne (GC) & Ręczne / Smart pointers \\
\hline
Architektura & GameObject-Component & Actor-Component \\
\hline
Natywne wsparcie 2D & Tak & Nie (symulowane) \\
\hline
Kod źródłowy & Częściowo dostępny & Pełny dostęp \\
\hline
Rozmiar pustego projektu & $\sim$100 MB & $\sim$1-2 GB \\
\hline
Krzywa uczenia & Łagodna & Stroma \\
\hline
Główne zastosowania & Mobile, indie, 2D & AAA, FPS, fotorealizm \\
\hline
\end{tabular}
\end{table}

\subsection{Uzasadnienie wyboru do badań}

Wybór Unity i Unreal Engine jako przedmiotu porównania pozwala na analizę dwóch fundamentalnie różnych podejść do tworzenia gier:

\begin{enumerate}
    \item \textbf{Produktywność vs wydajność} -- C\# w Unity oferuje szybszy rozwój kosztem pewnego narzutu wydajnościowego, podczas gdy C++ w Unreal wymaga więcej pracy, ale zapewnia maksymalną kontrolę
    \item \textbf{Dostępność vs specjalizacja} -- Unity celuje w szeroki rynek z niskim progiem wejścia, Unreal koncentruje się na produkcjach premium
    \item \textbf{Elastyczność vs integracja} -- Unity pozwala na większą swobodę w doborze zewnętrznych narzędzi, Unreal oferuje bardziej zintegrowane rozwiązania
\end{enumerate}

Analiza tych dwóch silników dostarcza kompleksowego obrazu współczesnego stanu technologii do tworzenia gier i pozwala na sformułowanie praktycznych rekomendacji dla deweloperów.
