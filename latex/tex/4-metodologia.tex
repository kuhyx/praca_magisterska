\clearpage
\section{Metodologia badań i kryteria porównania}

\subsection{Założenia metodologiczne}
\subsubsection{Cel badań}
Głównym celem badań jest obiektywne porównanie wydajności i możliwości wybranych silników gier w kontrolowanych warunkach.

\subsubsection{Hipotezy badawcze}
\begin{enumerate}
    \item Silniki komercyjne oferują lepszą wydajność niż rozwiązania open source
    \item Kompleksowość funkcjonalności wpływa negatywnie na wydajność
    \item Łatwość użycia jest odwrotnie proporcjonalna do możliwości konfiguracji
\end{enumerate}

\subsection{Kryteria porównania}
\subsubsection{Wydajność}
\begin{itemize}
    \item Szybkość renderowania (FPS)
    \item Zużycie pamięci RAM
    \item Obciążenie procesora
    \item Zużycie pamięci karty graficznej
    \item Czas ładowania scen
\end{itemize}

\subsubsection{Funkcjonalność}
\begin{itemize}
    \item Wsparcie dla różnych typów renderingu
    \item Systemy fizyki
    \item Systemy audio
    \item Wsparcie dla VR/AR
    \item Możliwości skryptowania
\end{itemize}

\subsubsection{Użyteczność}
\begin{itemize}
    \item Intuicyjność interfejsu
    \item Jakość dokumentacji
    \item Dostępność tutoriali
    \item Wsparcie społeczności
    \item Czas potrzebny na naukę
\end{itemize}

\subsection{Środowisko testowe}
\subsubsection{Specyfikacja sprzętowa}
% Szczegółowy opis konfiguracji sprzętowej używanej do testów

\subsubsection{Specyfikacja oprogramowania}
% Wersje systemów operacyjnych, sterowników, silników

\subsection{Projekt testów}
\subsubsection{Scenariusze testowe}
% Opis konkretnych scenariuszy używanych do porównania

\subsubsection{Metryki i wskaźniki}
% Definicja mierzonych parametrów i sposobów ich pomiaru
