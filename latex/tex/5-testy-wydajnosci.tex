\clearpage
\section{Testy wydajności}

\subsection{Metodyka przeprowadzania testów}
\subsubsection{Przygotowanie środowiska testowego}
% Opis procesu przygotowania spójnego środowiska dla wszystkich silników

\subsubsection{Standaryzacja warunków testowych}
% Zapewnienie porównywalności wyników między różnymi silnikami

\subsection{Test renderowania 2D}
\subsubsection{Założenia testu}
% Opis scenariusza testowego dla grafiki 2D

\subsubsection{Wyniki pomiarów}
% Tabele i wykresy przedstawiające wyniki dla każdego silnika

\subsubsection{Analiza wyników}
% Interpretacja uzyskanych danych

\subsection{Test renderowania 3D}
\subsubsection{Scenariusz podstawowy}
% Test renderowania prostej sceny 3D

\subsubsection{Scenariusz zaawansowany}
% Test złożonych scen z wieloma obiektami, oświetleniem, cieniami

\subsubsection{Porównanie wyników}
% Analiza porównawcza wydajności renderowania 3D

\subsection{Test systemów fizyki}
\subsubsection{Symulacja kolizji}
% Testy wydajności przy różnej liczbie obiektów fizycznych

\subsubsection{Symulacja płynów i cząstek}
% Testy zaawansowanych systemów fizyki

\subsection{Test zużycia zasobów systemowych}
\subsubsection{Zużycie pamięci RAM}
% Pomiary zużycia pamięci operacyjnej

\subsubsection{Obciążenie procesora}
% Analiza wykorzystania CPU

\subsubsection{Zużycie pamięci GPU}
% Pomiary zużycia pamięci karty graficznej

\subsection{Test wydajności na różnych platformach}
\subsubsection{Testy na PC (Windows/Linux)}
% Porównanie wydajności na systemach desktop

\subsubsection{Testy na urządzeniach mobilnych}
% Analiza wydajności na platformach mobilnych (jeśli dotyczy)

\subsection{Podsumowanie wyników testów wydajności}
% Zestawienie wszystkich uzyskanych wyników w formie tabelarycznej i graficznej
